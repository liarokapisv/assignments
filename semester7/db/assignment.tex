\documentclass{assignment}

\class{Βάσεις Δεδομένων}
\assignment{2η Σειρά Ασκήσεων}

\begin{document}
\maketitle
\subsection*{Άσκηση 1}
\begin{enumerate}[(1)]
\item Λάθος, διότι υπάρχουν και οι εξαρτήσεις $BCi\rightarrow C, BC\rightarrow B$.
\item Σωστό, διότι υπάρχει η εξάρτηση $BC\rightarrow D$ και η εξάρτηση $D\rightarrow EG \Rightarrow  BCDEG\subset BC^{+}$. Αντίστοιχα, $C\rightarrow A$. Τελικά όλα τα γνωρίσματα ανοίγουν στο $BC^{+}$.
\item Σωστό, διότι καμία σχέση δεν έχει ως δεξί μέλος τα $A,C,AC$.
\item Σωστό, διότι όλα τα γνωρίσματα του $R$ περιλαμβάνονται στο $BC$.
\item Λάθος, διότι μπορούμε να βγάλουμε το $A$, να κρατήσουμε τα $BC$ και να συνεχίσουμε να έχουμε όλα τα γνωρίσματα.
\item Λάθος, διότι και το $CG$ είναι υποψήφιο κλειδί.
\item Σωστό, διότι υπάρχουν οι εξαρτήσεις $C\rightarrow A$, $CG\rightarrow BD  \Rightarrow  ABCDG\subset CG^{+}$, $D\rightarrow EG$ και επομένως το $CG$ είναι υποψήφιο κλειδί.
\item Σωστό, διότι υπάρχουν οι εξαρτήσεις $C\rightarrow A$ , $ACD\rightarrow B , D\rightarrow EG , C\rightarrow A$ και επομένως το $CD$ είναι υποψήφιο κλειδί.
\item Λάθος, διότι καμία σχέση δεν έχει στο δεξί μέλος τα $A,G$.
\item Λάθος, διότι έστω και αν είναι όντως κλειδί, τα ίδια γνωρίσματα θα περιλαμβάνονται στο $E$ αν το αφαιρέσουμε. 
\end{enumerate}

\newpage
\subsection*{Άσκηση 2}


\selectlanguage{english}
\begin{center}
\begin{tabular}{|c|c|c|c|}
    \hline
    A & B & C & D \\ \hline
    a1 & b2 & c3 & d1 \\ \hline
    a1 & b2 & c3 & d3 \\ \hline
    a2 & b3 & c2 & d2 \\ \hline
    a3 & b4 & c3 & d1 \\ \hline
\end{tabular}
\end{center}
\selectlanguage{greek}

\begin{enumerate}

\item
Αν παρατηρήσουμε όλες τις γραμμές στις οποίες το $A$ παραμένει σταθερό, μπορούμε να συμπεράνουμε πως και το $B$ και το $C$ παραμένουν σταθερά. 
Mε αυτό τον τρόπο προκύπτουν οι εξαρτήσεις $A \rightarrow B, A \rightarrow C$ και$B \rightarrow A, B \rightarrow C$.
Δεν συμβαίνει το ίδιο στην τρίτη στήλη καθώς κανένα άλλο γνώρισμα δεν μένει σταθερό στις γραμμές που το $C$ παραμένει σταθερό. 
Από την τέταρτη στήλη παρατηρούμε πως υπάρχει η εξάρτηση $D \rightarrow C$. 

\selectlanguage{english}
\begin{center}
\begin{tabular}{|c|c|c|c|}
    \hline
    A & B & C & D \\ \hline
    a1 & b2 & c3 & d1 \\ \hline
    a1 & b2 & c3 & d3 \\ \hline
    a2 & b3 & c2 & d2 \\ \hline
    a3 & b4 & c2 & d1 \\ \hline
\end{tabular}
\end{center}
\selectlanguage{greek}

\item 
Παρατηρούμε πως όλες οι σχέσεις που εμπεριέχουν τα $A,B$ στο δεξί μέλος, παραμένουν σταθερές. Παρατηρούμε πως στην τρίτη στήλη το $C$ μένει σταθερό όταν τα $A,B$ παραμένουν σταθερά. 
Έτσι έχουμε τις εξαρτήσεις $C \rightarrow A, C \rightarrow B$. 
Επίσης όταν το $D$ παραμένει σταθερό, όλα τα άλλα μεγέθη μεταβάλλονται και επομένως δεν υπάρχει εξάρτηση με το $D$ ως δεξί γνώρισμα.

\end{enumerate}

\newpage
\subsection*{Άσκηση 3}
\begin{enumerate}[(1)]
\item $A\rightarrow D,A \rightarrow B ,C\rightarrow A$ 
\begin{enumerate}[(a)]
\item Το $C$ είναι το μοναδικό κλειδί αφού υπάρχει η εξάρτηση $C \rightarrow ABCD$ και το $ABC$ δεν αποτελεί υπερκλείδι αφού δεν ισχύει η αντίστροφη εξάρτηση.
\item Έχουμε πως το $R$ είναι $2NF$ μιας και όλα τα μη πρωτεύοντα γνωρίσματα δεν εξαρτώνται από το $C$. Επιπλέον, δεν το $R$ δεν είναι $3NF$ διότι το $D$ εξαρτάται μεταβατικά από το $C$ και δεν είναι πρωτεύον γνώρισμα.
\item Αν κάνουμε σε κάθε σχέση μία αποσύνθεση, θα καταλήξουμε στις σχέσεις $R_1 = {A,D}, R_2 = {A, B}, R_3 = {C,A}$. 
\end{enumerate}
\item $C\rightarrow A, B\rightarrow D$
\begin{enumerate}[(a)]
\item Αντίστοιχα με το προηγούμενο ερώτημα, το κλειδί είναι το $BC$
\item Έχουμε πως τo $R$ είναι $1NF$ διότι το Α δεν είναι πρωτεύον γνώρισμα και υπάρχει η $C\rightarrow A $.
\item Αντίστοιχα με πριν, έχουμε τις αποσυνθέσεις $R_1 ={A,C}$ (όχι BNFC),$R_2 ={B,D}$,και $R_3 ={B,C}$.
\end{enumerate}

\end{enumerate}
\subsection*{Άσκηση 4}

\begin{enumerate}[(A)]
\item Αφού το μέγεθος του \en{page} είναι 1024 \en{bytes}, και το δέντρο είναι τάξης $p$, για εσωτερικούς κόμβους θα ισχύει πως $p*12 + (p-1)*45 \le 1024$. Αφού το δέντρο είναι πυκνό θα
    έχουμε $p=18$. Για τα φύλλα θα ισχύει $p_\text{\en{leaf}} * (12 + 45) + 12 \le 1024$ και επομένως $p_\text{\en{leaf}} = 17$. 
Συνολικά έχουμε 11000 εγγραφές και επομένως θα έχουμε $\left\lfloor\frac{11000}{17}\right\rfloor = 648$ φύλλα.
Στο επόμενο επίπεδο θα έχουμε $\left \lfloor\frac{648}{18}\right\rfloor = 36$ κόμβους, στο επόμενο 2 και θα έχουμε και την κορυφή. Έτσι συνολικά έχουμε 4 επίπεδα.

\item \en{i}.Για το δέντρο των έμμεσων δεικτών θα έχουμε ακολουθώντας την παραπάνω διαδικασία, 100 φύλλα και 1 κορυφή. Έτσι συνολικά θα έχουμε, $648+36+1+1$ \en{blocks} από το αρχικό δέντρο και $100+1$ απο το δεύτερο. Τελικά θα έχουμε 878 \en{blocks}.  
\en{ii.}. Ο αριθμός προσπέλασης σελίδας θα είναι όσες και οι μεταβάσεις επιπέδων. Έτσι θα έχουμε 4 μεταβάσεις για το 1ο δέντρο και 2 μεταβάσεις για το 2ο. Τελικά θα έχουμε συνολικά 6 προσπελάσεις.




\end{enumerate} 

\end{document}


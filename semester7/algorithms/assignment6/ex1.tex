\subsection*{Άσκηση 1}

\begin{enumerate}[(1)]

\item Μπορούμε να κρατήσουμε μία ουρά ελεγμένων κόμβων και άλλη μία ορίου. Αρχίζοντας με τον κόμβο $u_1$, για κάθε μη-ελεγμένο γείτονα του, αντιστοιχούμε μία απόσταση και τον βάζουμε στην ουρά της επόμενης επανάληψης. Μετά διασχίζουμε μία φορά την ουρά της επόμενης επανάληψης και κοιτάμε άμα αντιστοιχούν οι αποστάσεις. Αν όχι ο αλγόριθμος απαντάει αρνητικά διαφορετικά βάζουμε τον κόμβο στο σύνολο ελεγμένων κόμβων αλλα και στην ουρά των αρχικών κόμβων. Αδειάζουμε την ουρά της επόμενης επανάληψς και επαναλαμβάνουμε την παραπάνω διαδικασία για κάθε κόμβο στην ουά αρχικών κόμβων. Στο τέλος αν υπάρχει κόμβων που δεν ανοίκει στο σύνολο ελεγμένων ο αλγόριθμος απαντάει αρνητικά αλλιώς θετικά.

\item Για κάθε ζεύγος $u-v$ υπάρχει μόνο μία πιθανή περίπτωση καλύτερης διαδρομής (άμα περνάει απο το $x-y$) οπότε θα έχουμε $d'(u,v) = min{d(u,x)+d(x,y)+d(y,v), d(u,y)+d(x,y)+d(x,v), d(u,v)}$. Κάνουμε την
    παραπάνω ανάθεση σε κάθε ζεύγος άρα η συνολική πολυπλοκότητα είναι $|V|^2$.

\item Αυτή η περίπτωση δεν σχετίζεται με την προηγούμενη περίπτωση. Η μείωση ενός κόμβου συνεπάγεται σε μία το πολύ περίπτωση καινούργιας καλύτερης διαδρομής την οποία και γνωρίζουμε. Αν αυξηθεί ένα ζεύγος ωστόσο δεν γνωρίζουμε τις εναλλακτικές επιλογές καλύτερων διαδρομών οπότε και πρέπει να τις ξαναυπολογίσουμε.


\end{enumerate}

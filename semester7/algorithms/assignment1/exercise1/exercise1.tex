\newpage
\subsection*{Άσκηση 1}

\subsubsection*{Μέρος α}

\begin{itemize}
    \item $g_1(n) = n^4 = \Theta(n^4)$
    \item $g_2(n) = 2^{\log^{3}n} = \Theta(n^{\log^{2}n})$
    \item $g_3(n) = \frac{\log(10n!)}{\log^9n} = \Theta(\frac{\log(n!)}{\log^9n}) = \Theta(\frac{n\log n}{\log^9n}) =  \Theta(\frac{n}{\log^8n})$
    \item $g_4(n) = n3^{4^{5^6}} = \Theta(n)$
    \item $g_5(n) = \Theta(n)$
    \item $g_6(n) = \sum_{k=0}^{n}k^3 = n^4 = \Theta(n^4)$
    \item $g_8(n) = \sqrt{n!} = \Theta(\sqrt{n!})$
    \item $g_9(n) = \sum_{k=0}^n k2^{-k} = 2 = \Theta(1)$
    \item $g_{10}(n) = \frac{n}{\log^{10}n} = \Theta(\frac{n^5}{\log^{10}n})$
    \item $g_{11}(n) = n \sum_{k=0}^{n} \binom{n}{k} = n 2^n = \Theta(n2^n)$
    \item $g_{12}(n) = O(n)$
    \item $g_{13}(n) = \binom{2n}{n/4} = \Theta(\binom{8n}{n}) = O(n!)$
    \item $g_{14}(n) = \sum_{k=0}^n k2^k = \Theta(n2^n)$
\end{itemize}


Για το $g_{12}(n)$ έχουμε, $n\log2 \le \log\binom{2n}{n} \le n\log(2e) $, άρα $g_{12}(n) = \Theta(n)$.\\
Η $g_5(n)$ έχει άνω φράγμα μικρότερο απο $O(g_{12})$

Τελικά έχουμε την εξής ταξινόμηση,
\begin{align*} & O(g_9) \le O(g_5) \le O(g_3) \le O(g_4) \le O(g_{12}) \le O(g_1) \le O(g_6) \\
               & O(g_{10}) \le O(g_7) \le O(g_2) \le O(g_{11}) \le O(g_{14}) \le O(g_{13}) \le O(g_8)
\end{align*}

\subsubsection*{Μέρος β}

\begin{enumerate}[(i)]

    \item \begin{align*} 
            T(n) &= 3T(\frac{n}{4}) + n\log\log^7n \\
                 &= 3(3T(\frac{n}{4^2}) + \frac{n}{4}\log\log^7(\frac{n}{4})) + n\log\log n \\
                 &= 3^2T(\frac{n}{4^2}) + 3\frac{n}{4^1}\log\log^7(\frac{n}{4^1}) + n\log\log n \\
                 &= 3^2T(\frac{n}{4^2}) + \sum_{k=1}^{1} 3^k\frac{n}{4^k}\log\log^7\frac{n}{4^k} \\
                 &= 3^2(3T(\frac{n}{4^3}) + \frac{n}{4^2}\log\log^7(\frac{n}{4^2})) + \sum_{k=1}^{1} 3^k\frac{n}{4^k}\log\log^7\frac{n}{4^k} \\
                 &= 3^3T(\frac{n}{4^3}) + \sum_{k=1}^{2} 3^k\frac{n}{4^k}\log\log^7\frac{n}{4^k} \\
                 &...\\
                 &= 3^{\log_4n}T(1) + \sum_{k=1}^{\log_4n}3^k\frac{n}{4^k}\log\log^7\frac{n}{4^k}
           \end{align*}

            Επίσης,
            \begin{align*}
            A(n) &= \sum_{k=1}^{\log_4n}3^k\frac{n}{4^k}\log\log^7\frac{n}{4^k} \\
                 &= 7\sum_{k=1}^{\log_4n}3^k\frac{n}{4^k}\log\log\frac{n}{4^k} \\
                 &= 7\sum_{k=1}^{\log_4n}3^k\frac{n}{4^k}\log(\log n - \log{4^k}) \\
                 &= 7\sum_{k=1}^{\log_4n}3^k\frac{n}{4^k}\log(\log n - \frac{\log_4{4^k}}{\log_4{2}}) \\
                 &= 7\sum_{k=1}^{\log_4n}3^k\frac{n}{4^k}\log\log n - 7\sum_{k=1}^{\log_4n}3^k\frac{n}{4^k}\log\frac{\log_4{4^k}}{\log_4{2}} \\
                 &= 7n(\log \log n \sum_{k=1}^{\log_4n} \frac34^k - \sum_{k=1}^{\log_4n}\frac34^k\log\frac{k}{\log_42})
            \end{align*}

            Ο δεύτερος όρος του $A(n)$ είναι $\Theta(1)$ και επομένως $A = \Theta(n\log\log n)$. Άρα $T = \Theta(n\log\log n)$.

    \item Από \en{Master Theorem}, $T = \Theta(n\log n)$
    \item Από \en{Master Theorem}, $T = \Theta(n^{\log_4{9}})$
    \item $ T(n) = \sum_{k=1}^n + T(1) = \log(\prod_{k=1}^n) + T(1) = \log(n!) + T(1) = \Theta(n\log n)$
\end{enumerate}





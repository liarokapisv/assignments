\newpage
\subsubsection*{Άσκηση 3}

(a) Η ταξινόμηση μπορεί να γίνει με χρήση \en{couting sort} ο οποίος έχει πολυπλοκότητα $O(n+k)$, $O(k)$ για την αρχικοποίηση των κουβάδων και $O(n)$ για την εισαγωγή και εξαγωγή των στοιχείων απο αυτούς.
Το κάτω φράγμα δεν ισχύει γιατι ο \en{counting sort} δεν είναι αλγόριθμος σύγκρισης.

(b) Μπορούμε να χρησιμοποιήσουμε έναν τροποποιημένο για $n$-αδικoύς αριθμούς \en{counting sort} που κάνει \en{sort} με βάση ένα συγκεκριμένο ψηφίο των δοθέντων αριθμών και έχει πολυπλοκότητα Ο(n+N).
Με τη χρήση αυτού και κάνοντας διαδοχικές ταξινομίσεις στους δοθέντες αριθμούς απο το \en{LSD} στο \en{MSD}, μπορούμε να ταξινομίσουμε τους αριθμούς με κόστος $O(n+N)*O(d) = O(N)$ όπου $d$ 
ο μέγιστος αριθμός ψηφίων των αριθμών και σταθερά.


\subsection*{Άσκηση 3}

Μπορούμε να λύσουμε το πρόβλημα με τον παρακάτω ψευδοκώδικα:

\selectlanguage{english}
\begin{minted}{text}

rents_ending_at_[n] : each date is assigned all rents ending at that date O(m) construction
dp[n] : the best gain for each date

dp[0] = 0
for date from 1 to n:
   dp[date] = dp[date-1]
   for every rent in rents_ending_at(date):
        dp[date] = max(dp[date], dp[start_date(rent)] + gain(rent))

\end{minted}
\selectlanguage{greek}
        
Ο εξωτερικός βρόγχος θα τρέξει $n$ φορές και ο εσωτερικός συνολικά $m$ φορές.
Μαζί με την αρχικοποίηση του πίνακα τελικών ημερομηνιών ο αλγόριθμος έχει πολυπλοκότητα $O(n+m)$


\newpage
\subsection*{Άσκηση 1}

\begin{enumerate}[(1.)]
    \item
        Παρατηρούμε πως ο μέγιστος αριθμός ζευγών είναι $\min($ Πομποί, Δέκτες$)$. Έστι μπορούμε με δύο μετρητές και με μία γραμμική διάσχηση να
        βρούμε τον μέγιστο αριθμό ζευγών πετυχαίνοντας γραμμική πολυπλοκότητα.

    \item 
        Έστω $C(i,j)$ το κόστος της βέλτιστης λύσης μέχρι και την $i$ κεραία με $j$ πομπούς που δεν έχουν αντιστοιχηθεί με δέκτες. Τότε έχουμε την παρακάτω αναδρομική σχέση:
        $C(i,j) = \min{C(i-1, j+1) + R_i, C(i-1,j-1) + T_i}$.
        Η λύση θα είναι το $C(n,0)$. Μπορούμε να υπολογίσουμε όλες τις $C(i,j)$ σε $O(n^2)$ χρόνο με δυναμικό προγραμματισμό.

\end{enumerate}





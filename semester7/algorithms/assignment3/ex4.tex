\newpage

\subsubsection*{Άσκηση 4}

1. Ο αλγόριθμος είναι \en{greedy} φύσεως, παίρνει ως είσοδο μία λίστα απο $x$-συντεταγμένες σπιτιών και την ακτίνα
των κεραιών και επιστρέφει μία λίστα απο $x$-συντεταγμένες κεραιών:

\selectlanguage{english}
\begin{minted}{haskell}
    antennaLocs :: [Int] -> Int -> [Int]
    antennaLocs = antennaLocs' (-2*radious-1)
        where antennaLocs' :: Int -> [Int] -> Int -> [Int]
              antennaLocs _ [] _ = []
              antennaLocs leftestCoord (hc : rest) radius 
                | hc - leftestCoord > 2 * radius = (hc + radius) : antennaLocs hc rest radius
                | otherwise = antennaLocs leftestCoord rest radius

\end{minted}
\selectlanguage{greek}

Ο παραπάνω αλγόριθμος έχει πολυπλοκότητα $O(n)$.

2. Ο αλγόριθμος του πρώτου μέρους βασίζεται στο ότι υπάρχει μία καθορισμένη αρχή στην διάταξη των σπιτιών.
Σε έναν κύκλο ωστόσο μπορεί να υπάρχει επικάλυψη της τελικής και της αρχικής κεραίας και έτσι δεν υπάρχει εύκολος τρόπος
επιλογής αρχής. Ένας τρόπος επίλυσης είναι να διαλέξουμε κάθε σπίτι ως αρχή και να διαλέξουμε αυτην για την οποία ελαχιστοποιείται 
ο αριθμός των κεραιών σύμφωνα με τον παραπάνω αλγόριθμο. Αυτός ο αλγόριθμος έχει πολυπλοκότητα $O(n^2)$.

\newpage
\subsubsection*{Άσκηση 1}

Ο αλγόριθμος θα παίρνει ως είσοδο ένα ζεύγος συνόλων κλειδιών και κλειδαριων και θα
δίνει ως έξοδο μία λίστα ζευγαριών κλειδιών-κλειδαριών αυξανόμενου μεγέθους.

\selectlanguage{english}

\begin{minted}{haskell}
sortKeysLocks :: (Set Key, Set Lock) -> [(Key, Lock)]
sortKeysLocks (Empty, Empty) = []
sortKeysLocks data =
    let
       pivot = choosePivot data
       lessThanPivot = keepSmaller pivot data
       greaterThanPivot = keepLarger pivot data
    in
       sortKeysLocks lessThanPivot ++ [pivot] ++ sortKeysLocks greaterThanPivot

choosePivot :: Set Key -> Set Lock -> (Key, Lock)
keepSmaller :: (Key, Lock) -> (Set Key, Set Lock) -> (Set Key, Set Lock)
keepLarger :: (Key, Lock) -> (Set Key, Set Lock) -> (Set Key, Set Lock)
\end{minted}

\selectlanguage{greek}

Εξετάζοντας τον αλγόριθμο βλέπουμε πως έχει την ίδια μορφή με τον \en{quicksort}. Αρκεί να αποδείξουμε
πως οι συναρτήσεις \en{choosePivot, keepSmaller} και \en{keepLarger} έχουν πολυπλοκότητα $O(n)$ και
η ανάλυση θα ανάγεται στην ανάλυση του \en{quicksort} που έχει γνωστή πολυπλοκότητα $O(nlogn)$ στη μέση περίπτωση.

Η \en{choosePivot} μπορεί να διαλέξει τυχαία ένα κλειδί και μετά να το δοκιμάσει σε κάθε κλειδαριά έτσι ώστε να βρεί την αντίστοιχη κλειδαριά.
Έχει προφανώς πολυπλοκότητα $O(n)$.

H \en{keepSmaller} φιλτράρει δύο φορές, μία φορά για τα κλειδιά και μία για τα λουκέτα. 
Αυτο για τα κλειδιά το επιτυγχάνει βρίσκοντας ποια χωράνε στην \en{pivot} κλειδαριά, ενώ
για τα λουκέτα δοκιμάζει σε ποια δε χωράει το \en{pivot} κλειδί. Η πολυπλοκότητα της παραπάνω διαδικασίας
είναι $O(n)$ αφού χρειάζεται να διασχισει γραμμικά μία φορά κάθε σύνολο $n$ στοιχείων.

H \en{keepLarger} λειτουργεί με τρόπο αντίστοιχο της \en{keepSmaller} και επομένως έχει επίσης
πολυπλοκότητα $O(n)$.

\newpage
\subsubsection*{Άσκηση 5}

Θεωρούμε $\alpha(k,n)$ η οποία επιστρέφει τα λιγότερα βήματα στη χειρότερη περίπτωση.
Προφανώς $\alpha(1,n) = n-1$.
Έστω πως υπάρχει ύψος $h(k,n)$ απο το οποίο παίρνουμε τον βέλτιστο αριθμό βημάτων.
Τότε θα πρέπει να ισχύει:

\begin{equation*}
    \alpha(k,n) = 1 + \max(\alpha(k-1, h(k,n)), \alpha(k, l-h(k,n)) )
\end{equation*}

\begin{center}

\hfill

(Θέωρημα 1)

Έστω $h$ τυχαίο βήμα. Αν το αυξήσω κατα ένα αυξάνω το πολύ κατα 1 τα βήματα της περίπτωσης που το ποτήρι σπάει.
Αν το μειώσω κατα 1, μειώνω το πολύ κατα 1 τα βήματα της περίπτωσης που το ποτήρι δεν σπάει. 

\hfill

(Θεώρημα 2)

    Οι δύο παράμετροι του $\max$ θα πρέπει είτε να είναι ίσες είτε να διαφέρουν κατα 1. Έστω πως δεν ισχύει και $\alpha(k-1, h(k,n)) = \alpha_1$ και
    $\alpha(k-1, n-h(k,n)) = \alpha_2$ με $|a_1-a_2| \ge 1$. Tότε μπορώ να διαλέξω $h'(k,n) = h(k,n) \pm 1$ έτσι ώστε να μειωθούν κατα 1 το πολύ τα βήματα
    της μέγιστης περίπτωσης και να αυξηθούν κατα 1 το πολύ τα βήματα της ελάχιστης. Αν μειωθεί έστω και μία απο τις παραμέρους, και αφού έχουμε θεωρήσει πως απέχουν τουλάχιστον κατα 2, 
    η $\max$ τιμή θα μειωθεί σίγουρα και επομένως θα έχουμε βρει καλύτερο βέλτιστο βήμα το οποίο είναι άτοπο. Αν δεν μειωθεί μπορούμε να συνεχίσουμε να αλλάζουμε το βήμα μέχρι να βρεθούμε
    στην προηγούμενη περίπτωση.

\hfill

(Θεώρημα 3)

    Θα υπάρχει $h_1(k,n)$ τέτοιο ώστε $\alpha(k,n) = 1 + \alpha(k-1, h_1(k,n))$ και $h_2(,n): \alpha(k,n) = 1 + \alpha(k, n-h_2(k,n))$. Απο θεώρημα 2
    ξέρουμε πως οι διαφορά των δύο παραμέτρον για κάθε $h(k,n)$ θα είναι είτε 0 έιτε 1. Άμα είναι 0 
    το παραπάνω είναι αυταπόδεκτο. Για την περίπτωση που η διαφορά είναι 1 τότε μπορώ να διαλέγω κατάλληλα διαδοχικά εναλλακτικά βήματα και η διαφορά θα μειώνεται κατα το πολύ 2. 
    Άμα μειωθεί κατα 0 τότε διαλέγω άλλο βήμα, άμα μειωθεί κατα 1 τότε οι δύο παράμετροι θα είναι ίδιοι και επομένως θα ισχύει το θεώρημα. 
    Άμα μειωθεί κατα 2 τότε οι δύο παράμετροι θα αλλάξουν τιμές και θα έχω βρει κατάλληλα βήματα.


\newpage

    1. Έστω πως $x$ ο βέλτιστος αριθμός βημάτων. Tότε απο θεώρημα 3 θα υπάρχουν $h_1(2,n), h_2(2,n):$
    $$
    \begin{gathered}
        \alpha(2,n) = 1 + \alpha(1, h_1(2,n)) = h_1(2,n) = x \\
        \alpha(2,n) = 1 + \alpha(2, n-h_2(2,n)) = x
    \end{gathered}
    $$

    Έστω πως $h_1(2,n) \neq h_2(2,n)$, τότε $h_2(2,n) < h_1(2,n)$. Έστω πως δοκιμάζοντας στη θέση $h_1(2,n)$, το ποτήρι δεν σπάει, τότε σύμφωνα με θεώρημα 2 θα πρέπει
    $\alpha(2,n-h_1(2,n)) = x-1$ ή $x-2$. Έστω πως δοκιμάζοντας στη θέση $h_2(2,n)$ το ποτήρι δεν σπάει. Τότε υποχρεωτικά θα πρέπει $\alpha(2,n-h_2(2,n)) = x-1$ διαφορετικά
    οι βέλτιστες κινήσεις δεν θα ήταν $x$. Όμως $\alpha(2,n-h_2(2,n)) \ge \alpha(2,n-h_1(2,n))$ αφού $h_2(2,n) < h_1(2,n)$ και επομένως $\alpha(2,n-h_1(2,n)) = x-1$.

    Απο την τελευταία σχέση αποκτάμε αλυσίδα βημάτων: $x, (x-1), (x-2), .., 2, 1$ η οποία είναι $x$ στο πλήθος ($\alpha(2,n) = x$).
    Έτσι εξασφαλίζοντας πως όλες οι θέσεις που παίρνουμε με τα διαδοχικά βήματα είναι μέσα στο εύρος μας, έχουμε:

    $$
    \begin{gathered}
        x + (x-1) + (x-2) + .. + 1 \le n \\
        \frac{x(x-1)}{2} \le n \\
        x = \ceil*{\frac{-1 + \sqrt{1+8n}}{2}}
    \end{gathered}
    $$

    Για $n=100$ έχουμε $x = 14$.

\end{center}

    2. Σαν μία απλή λύση μπορούμε να υλοποιήσουμε απευθείας την αναδρομή της $\alpha(k,n)$ χρησιμοποιόντας δυναμικό προγραμματισμό για να μην υπολογίζουμε όλα τα υποπροβλήματα.
       Στην γενική περίπτωση είναι δύσκολο να υπολογίσουμε το βέλτιστο βήμα οπότε καταφεύγουμε σε εξαντλητική γραμμική αναζήτηση του βέλτιστου βήματος:

       \selectlanguage{english}
       \begin{minted}{haskell}


           solveA : Int -> Int -> Int (Cached)
           solveA (1, n) = (n,1)
           solveA (k, 1) = (1,1)
           solveA (k, 0) = (0,0)
           solveA (k, n) = minimum [ max(1 + solveA(k-1, s), 1 + solveA(k, n - s)), s <- [1 .. n] ]

       \end{minted}
       \selectlanguage{greek}

       Κάθε υποπρόβλημα θα υπολογιστεί μόνο μία φορά ενώ θα χρειαστεί $n$ επαναλήψεις για να βρεί το κατάλληλο βέλτιστο βήμα.
       Ο αλγόριθμος μπορεί να τροποποιηθεί έτσι ώστε να αποθηκεύεται και το βέλτιστο βήμα.
       Έτσι θα έχει πολυπλοκότητα $O(kn^2)$.

\newpage

    2. Έστω $\beta(t,k)$ συνάρτηση που επιστρέφει το μέγιστο εύρος που μπορεί να καλυφθεί με $k$ ποτήρια και $t$ προσπάθειες.
    Ο παρακάτω αναδρομικός τύπος εκφράζει το γεγονός πως το μέγιστο εύρος θα είναι το άθροισμα των μέγιστων ευρών στις δύο περιπτώσεις θράυσης
    των ποτηριών. 

    $$ \beta(t,k) = \beta(t-1, k-1) + 1 + \beta(t-1, k) $$

    Χρησιμοποιούμε μία βοηθητική συνάρτηση $\gamma$:

    \begin{align*}
        \gamma(t, k) &= \beta(t,k+1) - \beta(t, k)\\
                     &= (\beta(t-1,k+1)-\beta(t-1,k)) - (\beta(t-1, k)-\beta(t-1,k-1))\\
                     &= \gamma(t,k) - \gamma(t,k-1)
    \end{align*}
    
    Μαζί με τις αρχικές της συνθήκες παρατηρούμε πως

    $$\gamma(t,k) = \binom{t}{k+1}$$

    Τώρα μπορούμε να εκφράσουμε την $\beta$ ώς άθροισμα της $\gamma$:

    \begin{align*}
        \beta(t,k) &= \sum_{i=1}^{k} (\beta(t,i) - \beta(t,i-1) \\
        \beta(t,k) &= \sum_{i=1}^{k} \gamma(t,i-1) \\
        \beta(t,k) &= \sum_{i=1}^{k} \binom{t}{i}
    \end{align*}

    Έτσι μπορούμε υπολογίζοντας το $\beta(k,k) - \beta(k-1,k)$ να υπολογίσουμε το ύψος απο το οποίο μπορούν να αρχίσουν οι συγκρίσεις.
    Αν υποθέσουμε $O(k)$ αλγόριθμο για τον υπολογισμού του διονυμικού, θα έχουμε $\frac{k(k-1)}{2} = O(k^2)$ πολυπλοκότητα. 


    


\newpage
\subsubsection*{Άσκηση 2}

Ο Αλγόριθμος θα παίρνει ως είσοδο ένα \en{puzzle}, τα όρια της περιοχής που μας ενδιαφέρει και την
τοποθεσία του απαγορευτικού τετραγνώνου και θα γεμίζει την περιοχή με κατάλληλα κομμάτια.

\selectlanguage{english}
\begin{minted}{haskell}

//checks if limits describe the base case
isTwoByTwo :: Limits -> Bool

//puts puzzle piece on given adjacent points
fillPoints :: Puzzle -> (Point, Point, Point) -> Puzzle

//splits limits to four quadrants
splitToQuadrants :: Limits -> (Limits, Limits, Limits, Limits)

//returns fake forbiddens for all quadtrants, first limits is 
//given forbidden's quadrant. The forbiddens are adjacent.
getMockForbiddens :: Point -> (Limits, Limits, Limits, Limits) 
                           -> (Limits, (Limits,Point), (Limits,Point), (Limits,Point))

fillPuzzle :: Puzzle -> Limits -> Point -> Puzzle
fillPuzzle puzzle limits forbidden 
    | isTwoByTwo limits = .. //base case
    | otherwise = 
        let
          quadrants = splitToQuadrants limits forbidden
          (limitsOfForbidden,
            (limitsa, forbiddena), 
            (limitsb, forbiddenb), 
            (limitsc, forbiddenc)) = getMockForbiddens forbidden quadrants
          filledOfForbbiden = fillPuzzle puzzle limitsOfForbidden forbidden
          andFilledA = fillPuzzle filledOfForbbiden limitsa forbiddena
          andFilledB = fillPuzzle andFilledA limitsb forbiddenb
          andFilledC = fillPuzzle andFilledB limitsc forbiddenc
        in
          putPiece andFilledC (forbiddena, forbiddenb, forbiddenc)
    
\end{minted}
\selectlanguage{greek}

Για \en{puzzle} Μεγέθους $2\times2$ ο αλγόριθμος προφανώς δουλεύει.
Αν δουλεύει για μεγέθους $2^i\times2^i$ ο αλγόριθμος θα δουλεύει για $2^{i+1}\times2^{i+1}$ αφού
το \en{puzzle} σπάει σε 4 κομμάτια μεγέθους $2^i\times2^i$ τα οποία γεμίζονται αναδρομικά.
Το κομμάτι που περιέχει το απαγορευμένο κομμάτι θα γεμίσει πλήρως εκτός απο αυτό,
ενώ στα άλλα τρία κομμάτια βάζουμε ψεύτικα απαγορευτικά κομμάτια στην κοινή τους γωνία.
Στο τέλος και τα άλλα τρία θα γεμίσουν πλήρως εκτός απο τα κομμάτια της κοινής τους γωνίας
τα οποία μπορούμε και να γεμίσουμε με ένα κομμάτι του \en{puzzle}. 

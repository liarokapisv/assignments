\documentclass{assignment}
\usepackage{array}
\usepackage{calc}
\usepackage{color}
\usepackage{parskip}
\usepackage{tikz}

\class{Αρχιτεκτονική Υπολογιστών}
\assignment{Σειρά Ασκήσεων 2}

\newcolumntype{L}[1]{>{\raggedright\let\newline\\\arraybackslash\hspace{0pt}}m{#1}}
\newcolumntype{C}[1]{>{\centering\let\newline\\\arraybackslash\hspace{0pt}}m{#1}}
\newcolumntype{R}[1]{>{\raggedleft\let\newline\\\arraybackslash\hspace{0pt}}m{#1}}

\newcommand\tikzmark[2]{%
    \tikz[remember picture, anchor=base] \node[inner sep=0,outer sep=1] (#1){#2};%
}

\newcommand\link[2]{%
\begin{tikzpicture}[remember picture, overlay, >=stealth, shift={(0,0)}]
    \draw[->] (#1) to (#2);
\end{tikzpicture}%
}


\begin{document}

\maketitle

\newpage\subsubsection*{Άσκηση 1}

\selectlanguage{english}

\begin{center}
    \setlength\extrarowheight{2pt}
    \tiny
    \begin{tabular}{ |@{\ } C{\widthof{L1:}} @{\ } R{\widthof{ADDI}} @{\ } R{\widthof{xt3,}} @{\ } R{\widthof{0(xt0),}} @{\ } R{\widthof{mt3}} @{\ }|@{\ } C{\widthof{WB}} @{\ }|@{\ } C{\widthof{WB}} @{\ }|@{\ } C{\widthof{WB}} @{\ }|@{\ } C{\widthof{MEM}} @{\ }|@{\ } C{\widthof{WM}} @{\ }|@{\ } C{\widthof{WB}} @{\ }|@{\ } C{\widthof{MEM}} @{\ }|@{\ } C{\widthof{MEM}} @{\ }|@{\ } C{\widthof{WB}} @{\ }|@{\ } C{\widthof{WB}} @{\ }|@{\ } C{\widthof{MEM}} @{\ }|@{\ } C{\widthof{WB}} @{\ }|@{\ } C{\widthof{WB}} @{\ }|@{\ } C{\widthof{MEM}} @{\ }|@{\ } C{\widthof{WB}} @{\ }|@{\ } C{\widthof{WB}} @{\ }|@{\ } C{\widthof{MEM}} @{\ }|@{\ } C{\widthof{MEM}} @{\ }|@{\ } C{\widthof{WB}} @{\ }|@{\ } C{\widthof{WB}} @{\ }|@{\ } C{\widthof{MEM}} @{\ }|@{\ } C{\widthof{WB}} @{\ }|@{\ } C{\widthof{WB}} @{\ }|@{\ } C{\widthof{MEM}} @{\ }|@{\ } C{\widthof{WB}} @{\ }|}
\hline
         & & & & & 1 & 2 & 3 & 4 & 5 & 6 & 7 & 8 & 9 & 10 & 11 & 12 & 13 & 14 & 15 & 16 & 17 & 18 & 19 & 20 & 21 & 22 & 23 & 24 & 25  \\
\hline
        L1: & LW & {\color{red} \$t0}, & 0(\$t3) & & IF  & ID  & EX & MEM & WB &  &  &  &  & & & & & & & & & & & & & & & & \\
\hline
        & LW & {\color{blue}\$t1}, & 0({\color{red} \$t0}) & & & IF & ID & XX & XX & EX & MEM & WB &  &  &  & & & & & & & & & & & & & & \\
\hline
        & LW & {\color{green}\$t2}, & 8(\$t0) & & &  & IF & XX & XX & ID & EX & MEM & WB &  &  &  & & & & & & & & & & & & & \\
\hline
        & ADD & {\color{cyan}\$t2}, & {\color{green}\$t2}, & {\color{blue}\$t1} & &  &  &  XX & XX  & IF & ID & XX & XX & EX & MEM & WB & & & & & & & & & & & & &  \\
\hline
        & ADD & {\color{yellow}\$t2}, & {\color{cyan}\$t2}, & \$t0   &  & &  &  XX & XX &  & IF & XX & XX & ID & XX & XX & EX & MEM & WB & & & & & & & & & & \\
\hline
        & SW & {\color{yellow}\$t2}, & 0(\$t3) & &  &  & &  XX & XX & &  &  XX & XX   & IF & XX & XX & ID & XX & XX & EX & MEM & WB & & & & & & & \\
\hline
        & ADDI & {\color{magenta}\$t3}, & \$t3, & -4  &  & &  &  XX & XX &  &  & XX & XX &  & XX & XX & IF & XX & XX & ID & EX & MEM & WB & & & & & & \\
\hline
        & BNE & \$t9, & {\color{magenta}\$t3}, & L1  &  &  & &  XX & XX &  &  & XX & XX &  & XX & XX &  & XX & XX & IF & ID & XX & XX & EX & MEM & WB & & &  \\
\hline
        & LW & \$t0, & 0(\$t3) & &  & &  &  XX & XX  &  && XX & XX &  & XX & XX &  & XX & XX &  &  &  &  &  & IF & ID & EX & MEM & WB \\
\hline
\end{tabular}
\end{center}


\selectlanguage{greek}

Τα \en{RAW dependancies} έχουν σημειωθεί με αντίστοιχα χρώματα και αντιμετωπίζονται με τον εξής τρόπο:

\begin{itemize}
    \item Στο {\color{red}κόκκινο} κάνουμε \en{stall} για τους κύκλους 4 και 5.
    \item Στα {\color{blue} μπλε} και {\color{green} πράσσινο} αρκεί να κάνουμε \en{stall} για τους κύκλους 8 και 9.
    \item Στο {\color{cyan} κυανό} κάνουμε \en{stall} για τους κύκλους 11 και 12.
    \item Στο {\color{yellow} κίτρινο} κάνουμε \en{stall} για τους κύλους 14 και 15.
    \item Στο {\color{magenta} φούξια} κάνουμε \en{stall} για τους κύκλους 18 και 19.
\end{itemize}
    
Η τελευταία σειρά είναι η πρώτη εντολή απο την δεύτερη επανάληψη. Αφού η δεύτερη επανάληψη αρχίζει στον 21ο κύκλο, όλες οι επαναλήψεις εκτός απο την τελευταία μπορούν να θεωρηθούν
πως ολοκληρώνονται σε 20 κύκλους ενώ η τελευταία σε 22 κύκλους. Αφού στην εντολή άλματος συγκρίνονται οι \en{\$t3} και \en{\$t9} και σε κάθε κύκλο η \en{\$t3} μειώνεται κατα 4 ενώ η \en{\$t9} μένει σταθερή,
συνολικά θα διεξαχθούν $(4096-2048)/4 = 512$ επαναλήψεις. Έτσι συνολικά απαιτούνται $511*20 + 22 = 10242$ κύκλοι.

\newpage\subsubsection*{Άσκηση 2}

\selectlanguage{english}

\begin{center}
    \setlength\extrarowheight{2pt}
    \tiny
    \begin{tabular}{ |@{\ } C{\widthof{L1:}} @{\ } R{\widthof{ADDI}} @{\ } R{\widthof{xt3,}} @{\ } R{\widthof{0(xt0),}} @{\ } R{\widthof{mt3}} @{\ }|@{\ } C{\widthof{WB}} @{\ }|@{\ } C{\widthof{WB}} @{\ }|@{\ } C{\widthof{WB}} @{\ }|@{\ } C{\widthof{MEM}} @{\ }|@{\ } C{\widthof{WB}} @{\ }|@{\ } C{\widthof{MEM}} @{\ }|@{\ } C{\widthof{MEM}} @{\ }|@{\ } C{\widthof{WB}} @{\ }|@{\ } C{\widthof{MEM}} @{\ }|@{\ } C{\widthof{MEM}} @{\ }|@{\ } C{\widthof{MEM}} @{\ }|@{\ } C{\widthof{MEM}} @{\ }|@{\ } C{\widthof{MEM}} @{\ }|@{\ } C{\widthof{MEM}} @{\ }|@{\ } C{\widthof{WB}} @{\ }|@{\ } C{\widthof{MEM}} @{\ }|@{\ } C{\widthof{WB}} @{\ }| }
\hline
         & & & & & 1 & 2 & 3 & 4 & 5 & 6 & 7 & 8 & 9 & 10 & 11 & 12 & 13 & 14 & 15 & 16 & 17 \\
\hline
        L1: & LW & {\color{red} \$t0}, & 0(\$t3) & & IF & ID & EX & \tikzmark{a}{MEM} & WB & & & & & & & & & & & &\\
\hline
        & LW & {\color{blue}\$t1}, & 0({\color{red} \$t0}) & & & IF & ID & XX & \tikzmark{b}{EX} & MEM & WB & & &  &  &  &  &  & & & \\
\hline
        & LW & {\color{green}\$t2}, & 8(\$t0) & & &  & IF & XX & ID & EX & \tikzmark{c}{MEM} & WB & &  &  &  &  & & & & \\
\hline
        & ADD & {\color{cyan}\$t2}, & {\color{green}\$t2}, & {\color{blue}\$t1} & &  &  &  XX & IF & ID & XX & \tikzmark{d}{EX} & MEM & WB  & &  & & & & & \\
\hline
        & ADD & {\color{yellow}\$t2}, & {\color{cyan}\$t2}, & \$t0   &  & &  &  XX &  & IF & XX & ID & \tikzmark{e}{EX} & MEM & WB &  &  &  & & &\\
\hline
        & SW & {\color{yellow}\$t2}, & 0(\$t3) & &  &  & &  XX &  & & XX & IF & ID   & \tikzmark{f}{EX} & MEM & WB &  &  & & &\\
\hline
        & ADDI & {\color{magenta}\$t3}, & \$t3, & -4  &  & &  &  XX &  &  & XX &  & IF & ID & \tikzmark{g}{EX} & MEM & WB &  & & & \\
\hline
        & BNE & \$t9, & {\color{magenta}\$t3}, & L1  &  &  & &  XX &  &  & XX &  &  & IF & ID & \tikzmark{h}{EX} & MEM & WB & & &  \\
\hline
        & LW & \$t0, & 0(\$t3) & &  & &  &  XX &  &  & XX &  &  &  & &  & IF & ID & EX & MEM & WB  \\
\hline
\end{tabular}
\link{a}{b}
\link{c}{d}
\link{d}{e}
\link{e}{f}
\link{g}{h}
\end{center}



\selectlanguage{greek}

Για τις προωθήσεις χρησιμοποιήσαμε το γεγονός πως οι εντολές \en{LW} και \en{SW} εμφανίζουν το αποτέλεσμά τους στο \en{MEM} στάδιο. Όλες οι άλλες εντολές έχουν θεωρηθεί πως εμφανίζουν το
αποτέλεσμά τους στο \en{EX} στάδιο.
Επαναλαμβάνοντας την διαδικασία που ακολουθήσαμε στην άσκηση 1, καταλήγουμε πως για τις 512 επαναλήψεις απαιτούνται $511*12 + 14 = 6146$ κύκλοι.

\newpage\subsubsection*{Άσκηση 3}

\selectlanguage{english}

\begin{center}
    \setlength\extrarowheight{2pt}
    \tiny
    \begin{tabular}{ |@{\ } C{\widthof{L1:}} @{\ } R{\widthof{ADDI}} @{\ } R{\widthof{xt3,}} @{\ } R{\widthof{0(xt0),}} @{\ } R{\widthof{mt3}} @{\ }|@{\ } C{\widthof{WB}} @{\ }|@{\ } C{\widthof{WB}} @{\ }|@{\ } C{\widthof{WB}} @{\ }|@{\ } C{\widthof{MEM}} @{\ }|@{\ } C{\widthof{MEM}} @{\ }|@{\ } C{\widthof{MEM}} @{\ }|@{\ } C{\widthof{MEM}} @{\ }|@{\ } C{\widthof{MEM}} @{\ }|@{\ } C{\widthof{MEM}} @{\ }|@{\ } C{\widthof{MEM}} @{\ }|@{\ } C{\widthof{MEM}} @{\ }|@{\ } C{\widthof{MEM}} @{\ }|@{\ } C{\widthof{WB}} @{\ }|@{\ } C{\widthof{MEM}} @{\ }|@{\ } C{\widthof{WB}} @{\ }| }
\hline
         & & & & & 1 & 2 & 3 & 4 & 5 & 6 & 7 & 8 & 9 & 10 & 11 & 12 & 13 & 14 & 15 \\
\hline
        L1: & LW & {\color{red} \$t0}, & 0(\$t3) & & IF & ID & EX & \tikzmark{a}{MEM} & WB & & & & & & & & & & \\
\hline
        & ADDI & \$t3, & \$t3, & -4  &  & IF & ID & EX & MEM & WB & & & & & & & & & \\
\hline
        & LW & {\color{blue}\$t2}, & 8({\color{red} \$t0}) & & &  & IF & ID & \tikzmark{b}{EX} & \tikzmark{c}{MEM} & WB &  &  &  &  & & & & \\
\hline
        & LW & {\color{green}\$t1}, & 0(\$t0) & & & & & IF & ID & EX & \tikzmark{d}{MEM} & WB & & &  &  &  &  & \\
\hline
        & ADD & {\color{cyan}\$t2}, & {\color{blue}\$t2}, & \$t0   &  & &  & & IF & ID & \tikzmark{e}{EX} & MEM & WB &  &  &  & & & \\
\hline
        & ADD & {\color{yellow}\$t2}, & {\color{cyan}\$t2}, & {\color{green}\$t1} & &  &  & &  & IF & ID & \tikzmark{f}{EX} & MEM & WB  & &  & & & \\
\hline
        & SW & {\color{yellow}\$t2}, & 4(\$t3) & &  &  & &   &  & & IF & ID & \tikzmark{g}{EX} & MEM & WB &  &  & & \\
\hline
        & BNE & \$t9, & \$t3, & L1  &  &  & &  &  &  &  & IF & ID & EX & MEM & WB & & &  \\
\hline
        & LW & \$t0, & 0(\$t3) & &  & &  &  &  &  & &  & & & IF & ID & EX & MEM & WB  \\
\hline
\end{tabular}

    \link{a}{b}
    \link{c}{e}
    \link{d}{f}
    \link{e}{f}
    \link{f}{g}
\end{center}

\selectlanguage{greek}

Για το παραπάνω διάγραμμα αλλάξαμε θέση στην εντολή \en{ADDI} και έτσι καταφέραμε να εξαλείψουμε το πρώτο \en{stall}. Προκειμένου να γίνει αυτή η αλλαγή χρειάστηκε να αλλάξουμε το \en{offset} της εντολής \en{SW}.
Αντιμεταθέτοντας τις δύο εντολές \en{ADD} αλλα και τις δύο τελευταίες \en{LW}, καταφέραμε να εξαλείψουμε και το δεύτερο \en{stall}. Με αυτές τις αλλαγές και επαναλαμβάνοντας για άλλη μία φορά την διαδικασία που ακολουθήσαμε στην άσκηση 1, προκύπτει πως για όλες τις επαναλήψεις απαιτούνται $511*10 + 12 = 5122$ κύκλοι.
\end{document}


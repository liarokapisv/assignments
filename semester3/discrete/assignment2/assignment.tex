\documentclass{assignment}

\renewcommand{\class}{Διακριτές Μέθoδοι για την Επιστήμη Υπολογιστών}
\renewcommand{\assignment}{2η Σειρά Ασκήσεων}

\begin{document}

\maketitle

\newpage\subsection*{Θέμα 1}

\begin{enumerate}

\item
Επιλέγουμε αυθαίρετα $n$ φυσικούς αριθμούς απο το σύνολο $\{1,2,3,\dots,2^n-3, 2^n-2\}$. Να δείξετε ότι μεταξύ των αριθμών που έχουμε επιλέξει υπάρχει πάντα ένα ζευγάρι όπου ο μεγαλύτερος από τους δύο αριθμούς είναι μικρότερος ή ίσος από το διπλάσιο του άλλου (π.χ. για $n=3$, αν επιλέξουμε τους αριθμούς 1, 3, 6, έχουμε ότι $6 \leq 2 \cdot 3$)
\item
Θεωρούμε μία ακολουθία $N$ θετικών ακεραίων η οποία περιέχει ακριβώς $n$ διαφορετικούς αριθμούς. 
Να δείξετε ότι αν $N \ge 2^n$, υπάρχουν τουλάχιστον δύο ή περισσότερες διαδοχικές θέσεις της ακολουθίας τέτοιες ώστε το γινόμενο των αντίστοιχων αριθμών να είναι ένα τέλειο τετράγωνο. Π.χ. στην ακολουθίας 7,5,3,5,3,7, όπου $n=3$ και $N=2^3$, το γινόμενο των έξι τελευταίων διαδοχικών θέσεων είναι τέλειο τετράγωνο.
\end{enumerate}

\subsubsection*{Λύση}

\begin{enumerate}[(a)]

\item Έχουμε να διαλέξουμε $n$ αριθμούς. Ένας αριθμός $x$ ανήκει στη φωλιά $k$ αν $x = 2^k - m$ \\ 
για $2 <= m <= 2^{k-1}+1$ για $k$ απο 2 μέχρι $n$. Άρα έχω $n-1$ φωλιές.\\
Από αρχή περιστερώνα υπάρχουν δύο αριθμοί $x_1$, $x_2$ που ανήκουν στην ίδια φωλιά. \\
Άρα $x_1 = 2^k - m$ και $x_2 = 2^k - n$ για κάποιο $k$ και $m < n$.

\begin{gather*}
n-m \le 2^{k-1} - 1 \le 2^k - n\\
-m \le 2^k - 2n \\
2^k - m \le 2(2^k - n) \\
x_1 \le 2x_2
\end{gather*}

\item Έχουμε $2^n$ θέσεις και η θέση $k$ αντιστοιχίζεται σε ένα υποσύνολο των $n$ αριθμών, το οποίο
περιέχει τους αριθμούς που εμφανίζονται περιττές φορές μέχρι και τη θέση $k$. \\
Αν μία θέση $k$ αντιστοιχίζεται στο κενό σύνολο, τότε όλοι οι αριθμοί που εμφανίζονται μέχρι τη θέση $k$,
εμφανίζονται άρτιες φορές και επομένως το γινόμενό τους είναι τέλειο τετράγωνο. \\
Αν καμία θέση δεν αντιστοιχίζεται στο κενό σύνολο, τότε πρέπει να αντιστοιχίσουμε $2^n$ θέσεις 
σε $2^n - 1$ πιθανά υποσύνολα. Άρα θα υπάρχουν 2 θέσεις $k_1$ και $k_2$ για τις οποίες οι ίδιοι αριθμοί
θα εμφανίζονται περιττές φορές μέχρι αυτές. Επομένως οι αριθμοί που εμφανίζονται μεταξύ των $k_1+1$ και $k_2$,
θα εμφανίζονται άρτιες φορές και επομένως το αντίστοιχο γινόμενο θα είναι τέλειο τετράγωνο.\\
Άρα σε κάθε περίπτωση υπάρχει υποακολουθία της οποίας το γινόμενο είναι τέλειο τετράγωνο.

\end{enumerate}

\newpage\subsection*{Θέμα 2}

Θεωρούμε το γράφημα $G_1 = C_n * K_m$ που προκύπει από τη σύνδεση (\textlatin{join}) του κύκλου με 
$n \ge 3$ κορυφές με το πλήρες γράφημα με $m \ge 1$ κορυφές.

\begin{enumerate}[(a)]

\item 
Πόσες κορυφές και πόσες ακμές έχει το γράφημα $G_1$ (ως συνάρτηση των $n$ και $m$)?
\item
Για ποιες τιμές των $n$ και $m$ το γράφημα $G_1$ έχει κύκλο \textlatin{Euler}?
\item
Για ποιες τιμές των $n$ και $m$ το γράφημα $G_1$ έχει κύκλο \textlatin{Hamilton}?
\item
Ποιος είναι ο χρωματικός αριθμός του $G_1$ ?

\end{enumerate}

\subsubsection*{Λύση}

\begin{enumerate}[(a)]
\item
\begin{itemize}[label={}]
  \item $\displaystyle G_1^V = n + m$ 
  \item $\displaystyle G_1^E = n + \frac{m(m-1)}{2} + n*m$
\end{itemize}

\item
Για να υπάρχει κύλος \textlatin{Euler} θα πρέπει όλες οι κορυφές να είναι άρτιου βαθμού. 
Οι κορυφές του κύκλου $C_n$ είναι άρτιου βαθμού ενώ οι κορυφές του πλήρους γράφου $K_m$ είναι περιττού 
βαθμού αν το το $m$ είναι άρτιος και άρτιου βαθμού αν το $m$ είναι περιττό. Όταν συνδέσουμε τους δύο γράφους
για να φτιάξουμε τον $C_1$, κάθε κορυφή του $C_n$ θα έχει άρτιο βαθμό όταν το $m$ είναι άρτιος.
Κάθε κορυφή του $K_m$ θα έχει άρτιο βαθμό αν το $m$ είναι περιττό και το $n$ είναι άρτιο ή αν το $m$ 
είναι άρτιο και το $n$ είναι περιττό. Άρα το $C_1$ έχει κύκλο \textlatin{Euler} για $n$ περιττό και $m$ άρτιο.

\item
Ο κύκλος $C_n$ έχει κύκλο \textlatin{Hamilton}. Ο πλήρης γράφος $K_m$ έχει κύκλο \textlatin{Hamilton} αφού 
ισχύει η ικανή συνθήκη του \textlatin{Dirac}. Παίρνοντας τον κύκλο \textlatin{Hamilton} του $C_n$, βγάζοντας
μία ακμή, κάνοντας το ίδιο για τον $K_m$ και συνδέοντας τις τελικές κορυφές των δύο μεταξύ τους, προκύπτει ένας
κύκλος \textlatin{Hamilton} για όλο το $G_1$. Άρα κύκλος \textlatin{Hamilton} υπάρχει ανεξάρτητα απο τα $n$ και $m$.

\item
Έχουμε $\chi(C_n) = 2$ αν $n$ άρτιος και $\chi(C_n) = 3$ και $n$ περιττός. Επίσης έχουμε πως $\chi(K_m) = m$. Επειδή
κάθε κορυφή του $C_n$ συνδέεται με κάθε κορυφή του $K_m$, καμία κορυφή μεταξύ των δύο δεν μπορούν να έχουν ίδιο χρώμα
άμα θέλουμε να μην υπάρχουν γειτονικές κορυφές με το ίδιο χρώμα. Άρα όταν οι δύο γράφοι ενωθούν, το κομμάτι του κύκλου 
είναι χρωματισμένο με διαφορετικά χρώματα απο το κομμάτι του πλήρους γράφου. Άρα ο χρωματικός αριθμός του $G_1$ είναι
το άθροισμα των χρωματικών αριθμών των $C_n$ και $K_m$.

\end{enumerate}

\newpage\subsection*{Θέμα 3}

\begin{enumerate}[(a)]

\item
Να δείξετε ότι μπορούμε πάντα να προσανατολίσουμε όλες τις ακμές ενός μή κατευθυνόμενου συνεκτικού γραφήματος $G$ ώστε
για κάθε κορυφή $u$, ο προς-τα-έσω βαθμός και ο προς-τα-έξω βαθμός της $u$ είτε να έιναι ίσοι είτε να διαφέρουν κατα 1.

\item
Χρησιμοποιώντας μαθηματική επαγωγή στο $n$, να δείξετε ότι κάθε απλό γράφημα με $n \ge 3$ κορυφές και τουλάχιστον 
$\frac{(n-1)(n-2)}{2} +2$ ακμές έχει κύκλο \textlatin{Hamilton}. Να δείξετε ακόμη ότι για κάθε $n \ge 3$, υπάρχει 
απλό γράφημα με $n$ κορυφές και $\frac{(n-1)(n-2)}{2} + 1$ ακμές που δεν έχει κύκλο \textlatin{Hamilton}.

\end{enumerate}

\subsubsection*{Λύση}

\begin{enumerate}[(a)]

\item 
Το άθροισμα των βαθμών κάθε κόμβου είναι ίσο με το διπλάσιο του αριθμού ακμών. Άρα ο αριθμός των κόμβων με
περιττό βαθμό είναι άρτιος. Επομένως μπορούμε να χωρίσουμε το σύνολο αυτών των κόμβων σε ζεύγη οι κόμβοι των 
οποίων έπειτα συνδέονται με μία προστεθούμενη ακμή μεταξύ τους. Ο γράφος που προκύπτει θα έχει κόμβους άρτιου βαθμού και επομένως θα υπάρχει
κύκλος \textlatin{Euler} τον οποίο μπορούμε να ακολουθήσουμε και να προσανατολίσουμε τις ακμές με τέτοιο τρόπο
ώστε για κάθε κόμβο οι προς-τα-έσω βαθμοί ισούνται με τους προς-τα-έξω βαθμούς. Αφαιρώντας τις πρόσθετες ακμές έχουμε
προσανατολίσει τον αρχικό γράφο έτσι ώστε η διαφορά των προς-τα-έσω και προς-τα-έξω βαθμών κάθε κόμβου να είναι το πολύ 1.

\item 
Έστω γράφος $G(V, E)$ με $|E| \ge \binom{|V|-1}{2} + 2$. Αν ο $G$ είναι πλήρης τότε υπάρχει κύκλος \textlatin{Hamilton}.
An o $G$ δεν είναι πλήρης τότε $G'(V', E') = G - u - v$ όπου $u$ kαι $v$ μη γειτονικοί κόμβοι. Τότε:
\begin{equation*}
\binom{|V|-2}{2} \ge |E'| = |E| - (d(u) + d(v)) \ge \binom{|V|-1}{2} + 2 - (d(u) + d(v))
\end{equation*}

Απο τα παραπάνω προκύπτει
\begin{equation*}
d(u) + d(v) \ge |V|
\end{equation*}

Άρα απο το θεώρημα του \textlatin{Ore}, υπάρχει κύκλος \textlatin{Hamilton}.

Αν $|E|= \binom{|V|-1}{2} + 1$, τότε ο γράφος θα αποτελείται απο ένα πλήρες υπο-γράφημα $|V|-1$ κόμβων και έναν εξωτερικό κόμβο
συνδεδεμένο με έναν απο τους άλλους κόμβους. Προφανώς δεν μπορεί να υπάρχει κύκλος \textlatin{Hamilton} σε αυτήν την περίπτωση.
\end{enumerate}

\newpage\subsection*{Θέμα 4}

\begin{enumerate}[(a)]

\item
Έστω $n \ge 2$ θετικοί ακέραιοι $d_1d_2d_3\dots d_n$. Να δείξετε ότι $\sum_{i=1}^{n} d_i = 2(n-1)$ αν και μόνο αν
υπάρχει δέντρο T με n κορυφές και βαθμούς κορυφών $d_1d_2d_3\dots d_n$.

\item
Έστω απλό μη κατευθυνόμενο συνεκτικό γράφημα $G(V, E, w)$ με θετικά βάρη $w : E \rightarrow N^{*} $ στις ακμές. Μία ακμή
$e \in E$ καλείται {\it απαραίτητη} για το Ελάχιστο Συνδετικό Δέντρο (ΕΣΔ) του $G$ αν η αφαίρεση της οδηγεί σε αύξηση
του βάρους του ΕΣΔ, δηλ. αν $\text{ΕΣΔ}(G) < \text{ΕΣΔ}(G-e)$. Να δείξετε ότι μία ακμή $e \in E$ είναι απαραίτητη για το ΕΣΔ του $G$ 
αν και μόνο αν υπάρχει τομή $(S, V \backslash S)$ τέτοια ώστε η $e$ να είναι η {\it μοναδική ακμή ελάχιστου βάρους} που διασχίζει 
την $(S, V \backslash S)$, δηλ. για κάθε ακμή $e' = {u, v}$, με $u \in S$, $v \in V \backslash S$ και $e' \neq e$, έχουμε ότι
$w(e) < w(e')$.

\end{enumerate}

\subsubsection*{Λύση}

\begin{enumerate}[(a)]

\item 
Έστω πως $d_1d_2d_3\dots d_n$ θετικοί ακέραιοι τέτοιοι ώστε $\sum_{i=1}^{n} d_i = 2(n-1)$. 
Καταρχάς, υπάρχει γράφος με βαθμούς $d_1d_2d_3\dots d_n$, $n$ κόμβους και $n-1$ κορυφές, έστω όχι δέντρο. 
Τότε αναγκαστικά θα πρέπει να μην είναι συνεκτικός και επομένως θα έχει $k+1$ συνιστώσες και $k$ κύκλους (Η απόδειξη παραλείπεται). 
Τότε μπορούμε να βγάλουμε μία ακμή που ανοίκει σε έναν κύκλο μίας συνιστώσας, να βγάλουμε μία οποιαδήποτε ακμή μίας άλλης 
συνειστώσας και να eνώσουμε τους αντίστοιχους κόμβους των δύο συνιστωσών μεταξύ τους. Έτσι ο αριθμός των κόμβων, 
των ακμών και οι βαθμοί των κόμβων δεν αλλάζουν, όμως ο αριθμός των κύκλων και των συνιστωσών μειώνεται κατα ένα. 
Συνεχίζοντας αυτή τη διαδικασία καταλήγουμε στο επιθυμητό δέντρο. Το αντίστροφο προκύπτει απο την ιδιότητα των γράφων
να ισούται το άθροισμα των βαθμών των κόμβων με το διπλάσιο του αριθμού των ακμών.
 
\item

Έστω πως $e$ μία απαραίτητη ακμή του ΕΣΔ του $G(V, E)$. Αφού αφαιρέσω την $e$, προσθέτω μία άλλη $e'$ έτσι ώστε να εξακολουθεί να 
υπάρχει επικαλύπτον δέντρο. Θεωρώ την τομή $(S, V \backslash S)$, με το ένα άκρο των $e$ και $e'$ να ανήκει στο $S$ και το 
άλλο στο $V \backslash S$. Όμως με την $e'$, το δέντρο δεν είναι το ΕΣΔ άρα το βάρος του δέντρου αυξάνεται. Επομένως $w(e) < w(e')$.

Για το αντίστροφο, θεωρούμε πως υπάρχει τομή $(S, V \backslash S)$ με $e$, $e'$ να τη διασχίζουν, και με $w(e) < w(e')$. Τότε είτε έχουμε 
δύο ξένες συνεκτικές συνιστώσες και επομένως όχι δέντρο, είτε μία απο τις δύο ακμές ανήκει στο ΕΣΔ. Όμως μόνο μία απο τις δύο μπορεί να ανήκει 
στο δέντρο διαφορετικά θα είχαμε κύκλο. Άρα μόνο η $e$ μπορεί να ανήκει στο ΕΣΔ και είναι απαραίτητη γιατί άμα αφαιρεθεί και αντικατασταθεί
με την $e'$, το βάρος του ΕΣΔ θα αυξηθεί.

\end{enumerate}

\newpage\subsection*{Θέμα 5}

\begin{enumerate}

\item
Ενα επίπεδο γράφημα λέγεται εξωεπίπεδο αν μπορεί να σχεδιαστεί στο επίπεδο έτσι ώστε οι ακμές του να μην τέμνονται 
και όλες οι κορυφές του να βρίσκονται στην εξωτερική όψη. Να αποδείξετε ότι κάθε απλό εξωεπίπεδο γράφημα με $n$ κόμβους έχει 
το πολύ $2n-3$ ακμές. 

\item
Χρησιμοποιώντας μαθηματική επαγωγή στο $n$, να δείξετε ότι κάθε εξωεπίπεδο γράφημα έχει χρωματικό αριθμό μικρότερο ή ίσο του 3.

\end{enumerate}

\subsubsection*{Λύση}

\begin{enumerate}

\item
Έστω εξωεπίπεδος γράφος $G(V,E)$.
Ορίζω $G'(V', E')= G+u$ με $f'$ όψεις. Έστω $|E| > 2|V| - 3$.
Τότε έχουμε,
\begin{equation} |V'| = |V| + 1 \end{equation}
\begin{equation} |E'| = |E| + |V| > 3|V| - 3 = 3|V'| - 6 \label{eq2} \end{equation}

Σε κάθε ακμή αντιστοιχούν δύο πλευρές. Κάθε όψη έχει τουλάχιστον 3 πλευρές. Άρα,
\begin{align*} 
3f' &\le 2|E'|\\
2 + |E'| - |V'| &\le \frac{2|E'|}{3} \tag{Απο φόρμουλα \textlatin{Euler}} \\
|E'| - |V'|     &\le \frac{2|E'| - 6}{3}\\
|E'| &\le 3|V'| - 6
\end{align*}

Το οποίο σύμφωνα με τη (\ref{eq2}) είναι άτοπο.
Άρα $|E| \le 2|V| - 3$

\item
Έστω $G$ εξωεπίπεδος γράφος. 
Ορίζω $G'= G+u$. Για κάθε χρωματισμό του $G$, το $G'$ θέλει ένα παραπάνω χρώμα. Άρα $\chi(G') > \chi(G)$.
Όμως αν για κάποιο $G$, $\chi(G) \ge 4$ τότε $\chi(G') > 4$. Άτοπο σύμφωνα με το θέωρημα των τεσσάρων χρωμάτων.
Άρα $\chi(G) \le 3$.

\end{enumerate}

\newpage\subsection*{Θέμα 6}

\begin{enumerate}[(a)]

\item
Έστω $G$ και $\bar G$ ένα ζεύγος συμπληρωματικών γραφήματων με $n \ge 2$ κορυφές. Να δείξετε ότι $\chi(G)\chi(\bar G) \ge n$.

\item
Ένα γράφημα $G(V,E)$ ονομάζεται (χρωματικά) $k$-κρίσιμο εάν $\chi(G) = k$ και $\chi(G-u) < k$ για κάθε κορυφή $u \in V$.
Να δείξετε ότι για κάθε $k$-κρίσιμο γράφημα $G$, 
\begin{enumerate*}[(i)]
\item το $G$ είναι συνεκτικό, και
\item το $G$ έχει ελάχιστο βαθμό κορυφών $\delta(G)  \ge k-1$.
\end{enumerate*}
\end{enumerate}

\subsubsection*{Λύση}

\begin{enumerate}[(a)]

\item

Το $\alpha(G)$ είναι το μέγεθος του μέγιστου ανεξάρτητου υποσυνόλου των κόμβων του $G$. Στο $\bar G$ αυτό το σύνολο θα σχηματίζει το μέγιστο
πλήρες υπο-γράφημα και ο αριθμός των κόμβων αυτού είναι μικρότερος ή ίσος απο το χρωματικό αριθμό του $\bar G$.
Άρα, $\alpha(G) \le \chi(\bar G)$ και επομένως,
\begin{gather*}
\chi(G) \ge \frac{n}{\alpha(G)} \\
\chi(G) \alpha(G) \ge n \\
\chi(G) \chi(\bar G) \ge n
\end{gather*}

\end{enumerate}

\begin{enumerate}[({b}.i)]
\item 
Έστω πως ο γράφος δεν είναι συνεκτικός. Τότε θα αποτελείται απο δύο συνεκτικές συνιστώσες και ο χρωματικός αριθμός του θα ισούται 
με τον μέγιστο χρωματικό αριθμό των συνιστωσών. Όμως αν αφαιρέσουμε έναν κόμβο απο μία συνιστώσα που δεν έχει το μέγιστο
χρωματικό αριθμό, τότε ο χρωματικός αριθμός του γράφου δε θα αλλάξει. Άρα ο γράφος δεν θα είναι κρίσιμος. Άρα ο γράφος είναι συνεκτικός.

\item
Διαλέγω τον κόμβο $u$ του $G$ που έχει τον ελάχιστο βαθμό. Έχουμε $\chi(G) = k$. Επίσης γνωρίζουμε πως 
$\chi(G-u) \le k-1$ και επομένως μπορούμε να χρωματίσουμε τον $G-u$ με $k-1$ χρώματα. 
Έστω πως $\delta(G) \le k-2$, τότε ο $u$ έχει το πολύ $k-2$ γείτονες και επομένως μπορεί να προστεθεί 
στον $G-u$ και να χρωματιστεί με ένα απο τα $k-1$ χρώματα. Άρα θα μπορούσαμε να αποκτήσουμε ένα χρωματισμό του $G$ με $k-1$ 
χρώματα, το οποίο είναι άτοπο. Άρα $\delta(G) \ge k-1$.

\end{enumerate}


\end{document}

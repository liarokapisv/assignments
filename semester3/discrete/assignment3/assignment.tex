\documentclass{assignment}

\begin{document}

\createtitle{Διακριτές Μέθοδοι Για Την Επιστήμη Των Υπολογιστών}{3η Σειρά Ασκήσεων}

\begin{problem}{Θέμα 1}

Θεωρούμε 100 επιβάτες του προαστιακού, οι οποίοι έχουν επιβιβαστεί στο Αεροδρόμιο, και αποβιβάζονται σε 
κάποιους από τους επόμενους 12 σταθμούς (σε κάθε σταθμό αποβιβάζονται κανένας ή περισσότεροι επιβάτες). 
Με πόσους τρόπους μπορεί να συμβεί αυτό: 

\question{1} Αν θεωρήσουμε ότι οι επιβάτες δεν είναι διακεκριμένοι;
\question{2} Αν θεωρήσουμε ότι οι επιβάτες είναι διακεκριμένοι, και δεν παίζει ρόλο η σειρά αποβίβασης?
\question{3} Αν θεωρήσουμε ότι οι επιβάτες είναι διακεκριμένοι, και παίζει ρόλο η σειρά αποβίβασης.
\question{4} Αν θεωρήσουμε ότι οι επιβάτες είναι 45 άντρες και 55 γυναίκες, ότι τόσο οι άντρες μεταξύ
τους όσο και οι γυναίκες μεταξύ τους δεν είναι διακεκριμένοι, και ότι δεν παίζει ρόλο  η σειρά αποβίβασης
ανδρών και γυναικών?
\question{5} Αν θεωρήσουμε ότι και στο (4), με τη μόνη διαφορά ότι τώρα παίζει ρόλο η σειρά αποβίβασης 
ανδρών και γυναικών?
\question{6} Πόσοι είναι οι τρόποι αποβίβασης για τα ερωτήματα (1), (2) και (3), αν σε κάθε σταθμό 
κατεβαίνει τουλάχιστον ένας επιβάτης?

\solution

\answer
Μετράμε τρόπους να χωρέσουν 100 ίδια αντικείμενα (επιβάτες) σε 12 θέσεις (σταθμούς). \\
Αυτό μπορεί να συμβεί με $\binom{100 + 12 - 1}{12} = \binom{111}{12}$ τρόπους.

\answer
Σε κάθε άνθρωπο από τους 100 μπορώ να αντιστοιχίσω 12 θέσεις.\\
Άρα οι τρόπου να αποβιβαστούν οι επιβάτες είναι $12^{100}$.

\answer
Μετράμε τρόπους να χωρέσουν 100 διακεκριμένα αντικείμενα (επιβάτες) σε 12 θέσεις (σταθμούς) 
με τη διάταξη των αντικειμένων στην ίδια θέση να παίζει ρόλο. 
Υπάρχουν $\frac{(100 + 12 - 1)!}{(12-1)!} = \frac{111!}{11!}$ τρόποι να συμβεί αυτό.

\answer
Το πρόβλημα ανάγεται στα ανεξάρτητα προβλήματα του να μετρήσουμε τους τρόπους αποβίβασης
των γυναικών και των ανδρών. Το κάθε πρόβλημα ανάγεται στο να χωρέσουμε 55/45 άντρες/γυναίκες
σε 12 θέσεις. Άρα οι τρόποι αποβίβασης είναι $\binom{55+12-1}{55}\binom{45+12-1}{45} = \binom{66}{55}\binom{56}{45}$.

\answer
Μετράμε τρόπους να μοιραστούν 100 θέσεις σε δύο ομάδες των 45 ανδρών και 55 γυναικών.\\
Οι τρόποι να αποβιβαστούν οι επιβάτες είναι $\frac{100!}{45!55!}$.

\answer

Για το 
\begin{enumerate*}[(1)]
\item μετράμε τρόπους να τοποθετήσουμε 100 ίδια αντικείμενα (επιβάτες) σε
12 θέσεις (στάσεις) με κάθε στάση να έχει τουλάχιστον 1 αντικείμενο. 
Το πρόβλημα είναι αντίστοιχο με το να τοποθετήσουμε ένα από τα
αντικείμενα σε κάθε θέση και να μετρήσουμε τους τρόπους που χωράνε τα
υπόλοιπα 88 αντικείμενα στις 12 θέσεις. Άρα οι τρόποι αποβίβασης είναι
$\binom{12+88-1}{12}=\binom{99}{12}$.
Για το
\item ψάχνουμε αναδιατάξεις 100 διακεκριμένων αντικειμένων (επιβάτες)
σε 12 διακεκριμένες θέσεις με κάθε θέση να παίρνει τουλάχιστον ένα αντικείμενο.
Η Γ.Σ είναι η 
$(\frac{x^{1}}{1!} +\frac{x^{2}}{2!} + \frac{x^{3}}{3!} + \dots)^{12} 
=(e^x - 1)^{12}$ της οποίας ο όρος $\frac{x^{100}}{100!}$ έχει συντελεστή
$\sum\limits_{i=0}^{12}(-1)^i\binom{12}{i}(12-i)^{100}$. 
Για το 
\item ψάχνουμε τις αναδιατάξεις 100 διακεκριμένων αντικειμένων σε 12
διακεκριμένες θέσεις που παίζει ρόλο η σειρά. Για κάθε αναδιάταξη το πρόβλημα
ανάγεται στο να τοποθετήσω 100 ίδια άτομα σε 12 θέσεις. Άρα οι τρόποι 
αποβίβασης είναι $\binom{100+12-1}{12}100! = \binom{111}{12}100!$.
\end{enumerate*}
\end{problem}

\begin{problem}{Θέμα 2}
Θεωρούμε πρωτοβάθμια γλώσσα με ένα διμελές κατηγορηματικό σύμβολο $P$, την οποία 
ερμηνεύουμε σε σύνολο $A$ με $n$ (διακεκριμένα) στοιχεία. Να υπολογίσετε το
πλήθος των διαφορετικών διμελών σχέσεων στο $A$ οι οποίες (ως ερμηνεία του $P$)
ικανοποιούν τις παρακάτω προτάσεις:
\question{1} $\forall x P(x,x) \land \forall x \forall y (P(x,y)\rightarrow P(y,x))$
\question{2} $\forall x \forall y (P(x,y) \land P(y,x) \rightarrow x = y)$
\question{3} $\forall x\forall y\forall z 
               (P(x,y) \land P(y,z) \rightarrow P(x,z)) \land \forall x \forall y (P(x,y) \land P(y,x) \rightarrow x = y) \land\\
              \forall x \forall y (x \neq y \rightarrow P(x,y) \lor P(y,x))$
\end{enumerate}

\solution

\answer 
Το πλήθος των διμελών σχέσεων είναι ίσο με το πλήθος των συμμετρικών προς τη διαγώνιο πινάκων με διαγώνιο απο τιμές 1.
Ανάγεται στο πλήθος των συνδυασμών απο 0 και 1 στις $\frac{n(n-1)}{2}$ θέσεις που βρίσκονται κάτω/πάνω από
τη διαγώνιο. Άρα $2^{\frac{n(n-1)}{2}}$ πιθανές διμελές σχέσεις.

\answer
Σε ένα πίνακα γειτνίασης υπάρχουν 3 πιθανές τιμές κάθε ένα από τα  $\frac{n(n-1)}{2}$ ζεύγη $(P(x,y),P(y,x))$. Έτσι το 
πρόβλημα ανάγεται στα ανεξάρτητα προβλήματα της εύρεσης των συνδυασμών στις θέσεις της διαγωνίου και των συνδυασμών των 
παραπάνω ζευγών. Άρα οι διμελής σχέσεις είναι $2^n3^{\frac{n(n-1)}{2}}$.

\end{problem}

\begin{problem}{Θέμα 3}
\question{a} Ένα μάθημα παρακολουθείται από 500 φοιτητές και διδάσκεται σε 4 τμήματα από 4 διαφορετικούς καθηγητές (κάθε καθηγητής 
αναλαμβάνει ένα τμήμα εξ'ολοκλήρου και μπορεί να διαφοροποιηθεί ώς προς τον τρόπο εξέτασης). Να διατυπώσετε τη γεννήτρια συνάρτηση 
και να προσδιορίσετε τον όρο του οποίου ο συντελεστής δίνει τους διαφορετικούς τρόπους να χωριστούν οι φοιτητές σε τμήματα,
αν κάθε τμήμα πρέπει να έχει τουλάχιστον 50 και το πολύ 200 φοιτητές και:
\begin{enumerate}[1.]
\item οι φοιτητές είναι διακεκριμένοι και δεν έχει σημασία η σειρά με την οποία τοποθετούνται στα τμήματα.
\item οι φοιτητές είναι διακεκριμένοι και έχει σημασία η σειρά με την οποία τοποθετούνται στα τμήματα.
\end{enumerate}
\question{b} Να υπολογίσετε το πλήθος των τετραδικών συμβολοσειρών μήκους $n \ge 1$ στις οποίες το ψηφίο 0
εμφανίζεται τουλάχιστον μία φορά, το  ψηφία 1 έχει άρτιο πλήθος εμφανίσεων και το ψηφίο 2 έχει περιττό πλήθος
εμφανίσεων (δεν έχουμε περιορισμούς για το πλήθος των εμφανίσεων του ψηφίου 3).

\solution

\answer
Για το 
\begin{enumerate*}[1] 
\item η Γ.Σ είναι η $(\frac{x^{50}}{50!} + \frac{x^{51}}{51!} + \frac{x^{52}}{52!} + \dots + \frac{x^{200}}{200!})^4$ και μας
ενδιαφέρει ο συντελεστής του $\frac{x^{500}}{500!}$.
Για το
\item η Γ.Σ είναι η $(x^{50} + x^{51} + x^{52} + \dots + x^{200})^4$ και μας ενδιαφέρει ο συντελεστής του $\frac{x^{500}}{500!}$.
\end{enumerate*}

\answer
H Γ.Σ είναι η $(x + \frac{x^2}{2!} + \frac{x^3}{3!} + \dots)(1+\frac{x^2}{2!}+\frac{x^4}{4!}+\dots)
               (x+\frac{x^3}{3!}+\frac{x^5}{5!}+\dots)(1+x+\frac{x^2}{2!}+\dots)$ και μας 
ενδιαφέρει ο συντελεστής του $x^n$.
\end{problem}

\begin{problem}{Θέμα 4}

Θεωρούμε $n$ φοιτητές που απαντούν, ο καθένας ανεξάρτητα, σε μία ερώτηση. Για κάθε φοιτητή $i$, έστω $p_i \in [0,1]$ η πιθανότητα
να απαντήσει σωστά.

\question{1} Να διατυπώσετε τη γεννήτρια συνάρτηση και να προσδιορίσετε τον όρο του οποίου ο συντελεστής δίνει την πιθανότητα 
ακριβώς $k$ φοιτητές να απαντήσουν σωστά.

\question{2} Να διατυπώσετε τη γεννήτρια συνάρτηση και αν προσδιορίσετε τον όρο του οποίου ο συντελεστής δίνει την πιθανότητα
το πολύ $k$ φοιτητές να απαντήσουν σωστά.

\solution

\answer Η Γ.Σ είναι η $((1-p_1) + p_1x)((1-p_2) + p_2x)\dots((1-p_n) + p_n)$ και μας ενδιαφέρει ο συντελεστής του $x^k$.
\answer Η ακολουθία που προκύπτει για κάθε $i$ είναι το άθροισμα όλων των συντελεστών των όρων της παραπάνω ακολουθίας μέχρι
τον όρο $i$. Άρα η Γ.Σ είναι $\frac{((1-p_1) + p_1x)((1-p_2) + p_2x)\dots((1-p_n) + p_n)}{1-x}$ και ο συντελεστής που μας 
ενδιαφέρει είναι του $x^k$
\end{problem}

\end{document}

\documentclass{assignment}
\usepackage{mathrsfs}
\usepackage{array}
\usepackage{parskip}

\newcommand{\dropsign}[1]{\smash{\llap{\raisebox{-.5\normalbaselineskip}{$#1$\hspace{2\arraycolsep}}}}}%
\newcommand{\longdivline}{\cline{1-1} \\[\dimexpr-\normalbaselineskip+\jot]}

\begin{document}

\createtitle{Σήματα και Συστήματα}{3η \& 4η σειρά ασκήσεων}

\problem{Άσκηση 1}

Υπολογίστε τον αντίστροφο μετασχηματισμό $\mathscr{Z}$ και παρουσιάστε 
την περιοχή σύγκλισης (\en{ROC}) του μετασχηματισμού σε κάθε περίπτωση.

\begin{questions}

\question{1} $x_1[n] = \frac{u[n-1]}{n}$

\question{2} $x_2[n] = nu[n] +(2N-2n)u[n - (N+1)] - (2N - n)u[n-(2N+1)]$\\
          χρησιμοποιώντας το $z[n]=u[n]-u[n-N]$

%\question{3} $x_3[n] = 7\frac{1}{3}^n \cos(\frac{2\pi n}{6} + \frac{\pi}{4})$

\question{4} $x_4[n] = |n|2^{-|n|}$

\question{5} $x_5[n] = n(n+1)(n+2)3^{-n}u[n]$

\end{questions}
Υπολογίστε τον αντίστροφο μετασχηματισμός $\mathscr{Z}$ των:

\begin{questions}

\question{1} $\displaystyle W_1(z) = \frac{1-2z^{-1}}{z^{-1}-2}, |z| > \frac{1}{2}$ \\
             {\itshape Μέθοδος: Ολοκληρωτικά υπόλοιπα}
\question{2} $\displaystyle  X_1(z) = \sin(z)$, \en{ROC} περιέχει $\{|z|=1\}, X_2(z)=e^z+e^{\frac{1}{z}}, z \neq 0, X_3(z)=\ln(1-2z),|z| < \frac{1}{2}$\\
             {\itshape Μέθοδος: Δυναμοσειρά}
\question{3} $\displaystyle Y_1(z) = \frac{1}{(1-2z^{-1})(1-3z^{-1})(1+\frac{1}{2}z^{-1})}$, όπου $y[n]$ ευσταθές σήμα.\\
             $\displaystyle Y_2(z) = \frac{2z^4}{(-2+z)(-1+z)^2(-1+2z)}, |z| > 2$\\
             {\itshape Μέθοδος: Ανάλυση κλασμάτων}
\question{4} $\displaystyle Z_1(z) = \frac{1}{1-\frac{1}{3}z^{-1}}, |z| < \frac{1}{3}, Z_2(z) = \frac{1}{1-\frac{1}{3}z^{-1}}, |z| > \frac{1}{3}$ \\
             {\itshape Μέθοδος: Επαναλαμβανόμενη διαίρεση}

\end{questions}

\solution

\subsubsection*{Μετασχηματισμοί $\mathscr{Z}$} 

\begin{answers}

\answer
Γνωρίζουμε πως,
\begin{equation*}
x_1[n] = \frac{u[n-1]}{n}\\
\end{equation*}

Επίσης έχουμε,
\begin{gather*}
\frac{d\ln(1-x)}{dx} = -\frac{1}{1-x}\\
\frac{d\ln(1-x)}{dx} = -\sum_{k=0}^{\infty} x^k,\quad|x| < 1\\
\ln(1-x) = -\int\sum_{k=0}^{\infty} x^k \,dx + C ,\quad |x| < 1\\
\ln(1-x) = -\sum_{k=0}^{\infty} \frac{x^{k+1}}{k+1} \, dx, \quad \text{($C = 0$ για $x = 0$)}
\end{gather*}

Επομένως,
\begin{align*}
X_1(z) &= \mathscr{Z}\bigg\{x_1[n]\bigg\} \\
       &= \sum_{k=1}^{\infty}\frac{z^{-k}}{k} \\
       &= \sum_{k=0}^{\infty}\frac{z^{-k-1}}{k+1} \\
       &= \sum_{k=0}^{\infty}\frac{\left(\frac{1}{z}\right)^{k+1}}{k+1}\\
       &= ln\left(1-\frac{1}{z}\right),\quad \text{\en{ROC: }}\left|\frac{1}{z}\right| < 1 \Rightarrow |z| > 1
\end{align*}


\answer
Γνωρίζουμε πως, 
\begin{align*} x_2[n] &= n u[n] + 2Nu[n-(N+1)] - 2 n u[n-(N+1)] - 2 N u[n-(2N+1)] + n u[n-(2N+1)] \\
                        &= n u[n] + 2Nu[n-(N+1)] - 2 (n + (N+1) - (N+1)) u[n-(N+1)] \\ 
                        &\quad - 2Nu[n-(2N+1)] + (n+(2N+1)-(2N+1))u[n-(2N+1)] \\
                        &= n u[n] + 2Nu[n-(N+1)] - 2(n-(N+1))u[n-(N+1)] - 2(N+1)u[n-(N+1)] \\
                        &\quad - 2Nu[n-(2N+1)] + (n-(2N+1))u[n-(2N+1)] + (2N+1)u[n-(2N+1)] \\
                        &= n u[n] - 2u[n-(N+1)] - 2(n-(N+1))u[n-(N+1)] \\
                        &\quad + (n-(2N+1))u[n-(2N+1)] + u[n-(2N+1)]
\end{align*}

Επίσης έχουμε, 
\begin{gather*}
\mathscr{Z}\bigg\{u[n]\bigg\} = \frac{z}{z-1},\quad \text{\en{ROC: }} |z| > 1 \\
\mathscr{Z}\bigg\{nu[n]\bigg\} = \frac{z}{(z-1)^2} ,\quad \text{\en{ROC: }} |z| > 1\\
\mathscr{Z}\bigg\{u[n-k]\bigg\} = z^{-k}\frac{z}{z-1},\quad \text{\en{ROC: }} |z| > 1 \\
\mathscr{Z}\bigg\{(n-k)u[n-k]\bigg\} = z^{-k}\frac{z}{(z-1)^2},\quad \text{\en{ROC: }} |z| > 1
\end{gather*}

Επομένως,
\begin{align*} 
\mathscr{Z}\bigg\{x_2[n]\bigg\} &=  \mathscr{Z}\bigg\{nu[n]\bigg\} -2\mathscr{Z}\bigg\{u[n-(N+1)]\bigg\} 
                                                   -2\mathscr{Z}\bigg\{(n-(N+1))u[n-(N+1)]\bigg\} \\
       &\quad + \mathscr{Z}\bigg\{(n-(2N+1))u[n-(2N+1)]\bigg\} + \mathscr{Z}\bigg\{u[n-(2N+1)]\bigg\}\\
       X_2(z) &= \frac{z}{(z-1)^2} -2 z^{-(N+1)}\frac{z}{z-1} -2z^{-(N+1)}\frac{z}{(z-1)^2} 
              + z^{-(2N+1)}\frac{z}{(z-1)^2}+ z^{-(2N+1)}\frac{z}{z-1} ,\quad \text{\en{ROC: }} |z| > 1
\end{align*}


%\answer
%Γνωρίζουμε πως, 
%\begin{align*}
%x_3[n] &= 7\left(\frac{1}{3}\right)^n\cos\left(\frac{2\pi n}{6} + \frac{\pi}{4}\right) \\
%       &= \frac{7}{2}\left(\frac{1}{3}\right)^n\left(e^{j\left(\frac{2\pi n}{6} + \frac{\pi}{4}\right)} 
%                                                   + e^{-j\left(\frac{2\pi n}{6} + \frac{\pi}{4}\right)}\right)
%\end{align*}

\answer

Γνωρίζουμε πως,
\begin{gather*} x_3[n] = |n|2^{-|n|} = n\left(\frac{1}{2}\right)^{n}u[n] - n2^nu[-n-1]\end{gather*}

Επίσης έχουμε, 
\begin{gather*} 
\mathscr{Z}\bigg\{n\left(\frac{1}{2}\right)^nu[n]\bigg\} = 
  \frac{1}{2}\frac{z}{(z-\frac{1}{2})^2}, \quad\text{\en{ROC:} }|z| > \frac{1}{2} \\
\mathscr{Z}\bigg\{n2^nu[n]\bigg\} = 2\frac{z}{(z-2)^2}, \quad\text{\en{ROC:} }|z| < 2
\end{gather*}

Επομένως,
\begin{gather*}
X_3(z) = \frac{1}{2}\frac{z}{(z-\frac{1}{2})^2} + 2\frac{z}{(z-2)^2}, \quad\text{\en{ROC: }}\frac{1}{2} < |z| < 2 
\end{gather*}

\answer
Γνωρίζουμε πως,
\begin{equation*}x_5[n] = n(n+1)(n+2)3^{-n}u[n]\end{equation*}

Επίσης έχουμε,
\begin{gather*}
\mathscr{Z}\bigg\{3^{-n}u[n]\bigg\} = \frac{z}{z-\frac{1}{3}}, \quad\text{\en{ROC:} } |z| > \frac{1}{3} \\
\sum_{n=-\infty}^{\infty} 3^{-n}u[n]z^{-n} = \frac{z}{z-\frac{1}{3}} \\
-\sum_{n=-\infty}^{\infty} n3^{-n}u[n]z^{-n-1} = -\frac{3}{(1-3z)^2}, \quad\text{(Από παραγώγηση)} \\
\sum_{n=-\infty}^{\infty} n(n+1)3^{-n}u[n]z^{-n-2} = \frac{18}{(3z-1)^3}, \quad\text{(Από παραγώγηση)} \\
-\sum_{n=-\infty}^{\infty} n(n+1)(n+2)3^{-n}u[n]z^{-n-3} = -\frac{162}{(3z-1)^4}, \quad\text{(Από παραγώγηση)} \\
\sum_{n=-\infty}^{\infty} n(n+1)(n+2)3^{-n}u[n]z^{-n} = z^3\frac{162}{(3z-1)^4}, \quad\text{\en{ROC:} } |z| > \frac{1}{3} 
\end{gather*}

Επομένως,
\begin{equation*}
X_5(z) = z^3\frac{162}{(3z-1)^4}, \quad\text{\en{ROC:} } |z| > \frac{1}{3}
\end{equation*}

\end{answers}

\subsubsection*{Αντίστροφοι μετασχηματισμοί $\mathscr{Z}$}

\begin{answers}

\answer
Γνωρίζουμε πως, \begin{equation*}W_1(z) = \frac{1-2z^{-1}}{z^{-1}-2} = \frac{z-2}{1-2z}, \quad |z| > \frac{1}{2}\end{equation*}


\begin{equation*}
w_1[n] = \frac{1}{2\pi j}\oint_C W_1(z)z^{n-1}\,dz = \frac{1}{2\pi j}\oint_C \frac{z-2}{1-2z}z^{n-1}\,dz
\end{equation*}

\newpage
Παίρνω περιπτώσεις:

$n > 0$: Υπάρχει μόνο ο πόλος $\rho_1 = \frac{1}{2}$.
Η καμπύλη $C$ βρίσκεται στην περιοχή σύγκλισης άρα περικλύει τον πόλο.
Επομένως, 
\begin{align*}
w_1[n] &= Res\left[\frac{z-2}{1-2z}z^{n-1}, \frac{1}{2}\right] \\
       &= \lim_{z \to \frac{1}{2}}\left(z-\frac{1}{2}\right)\frac{z-2}{1-2z}z^{n-1} \\
       &= \lim_{z \to \frac{1}{2}}-\frac{1}{2}(z-2)z^{n-1} \\
       &= \frac{3}{2}\left(\frac{1}{2}\right)^n
\end{align*}

$n = 0$: Υπάρχουν οι πόλοι $\rho_1 = \frac{1}{2}$ και $\rho_2 = 0$.
Επομένως,
\begin{align*}
w_1[n] &= Res\left[\frac{1}{z}\cdot\frac{z-2}{1-2z}, \frac{1}{2}\right] + 
          Res\left[\frac{1}{z}\cdot\frac{z-2}{1-2z}, 0\right] \\
       &= \lim_{z \to \frac{1}{2}}\left(z-\frac{1}{2}\right)\frac{1}{z}\cdot\frac{z-2}{1-2z} +
          \lim_{z \to 0}\cdot\frac{z-2}{1-2z} \\
       &= \frac{3}{2} - 2 \\
       &= -\frac{1}{4}
\end{align*}

$n < 0$: Εφαρμόζουμε αλλαγή μεταβλητής $p = \frac{1}{z} \Rightarrow dp = -\frac{1}{z^2}dz,\quad |p| < 2$. Τότε,
\begin{equation*}
w_1[n] = \frac{1}{2\pi j}\oint_{C'}\frac{2p - 1}{p-2} p^{-n-1} \, dp
\end{equation*}

Υπάρχει μοναδικός πόλος $\rho_1 = 2$
Επίσης έχουμε η $C'$ εμπεριέχεται στην περιοχή σύγκλισης $|p| < 2$ και επομένως δεν περικλύει κανένα πόλο.
Επομένως,
\begin{equation*}
w_1[n] = 0
\end{equation*}

Τελικά:

\begin{equation*}
w_1[n] = \begin{cases}
    \frac{3}{2}\left(\frac{1}{2}\right)^n & n > 0 \\
    -\frac{1}{4} & n = 0 \\
    0 & n < 0
\end{cases}
\end{equation*}

\answer
Γνωρίζουμε πως,
\begin{align*}
X_3(z) &= \ln(1-x) \\
       &= -\sum_{k=1}^{\infty} \frac{(2z)^n}{n} \\
       &= \sum_{k=-\infty}^{\infty} -\frac{(2z)^n}{n} u[n-1] \\
       &= \sum_{k=-\infty}^{\infty} \frac{2^{-n}}{n} u[-n-1] z^{-n}
\end{align*}

Επίσης έχουμε,
\begin{equation*}
X_3(z) = \sum_{k=-\infty}^{\infty} x_3[n] z^{-n}
\end{equation*}

Επομένως,
\begin{equation*}
x_3[n] = \frac{2^{-n}}{n}u[-n-1]
\end{equation*}

\answer

Γνωρίζουμε πως,
\begin{gather*}
Y_1(z) = \frac{1}{(1-2z^{-1})(1-3z^{-1})(1+\frac{1}{2}z^{-1})} \\
\frac{Y_1(z)}{z} = \frac{z^2}{(z-2)(z-3)(z+\frac{1}{2})} \\
\frac{Y_1(z)}{z} = -\frac{8}{5}\cdot\frac{1}{z-2} + \frac{1}{35}\cdot\frac{1}{z+\frac{1}{2}} + \frac{18}{7}\cdot\frac{1}{z-3} \\
Y_1(z) = -\frac{8}{5}\cdot\frac{z}{z-2} + \frac{1}{35}\cdot\frac{z}{z+\frac{1}{2}} + \frac{18}{7}\cdot\frac{z}{z-3} \\
\end{gather*}

Επομένως,
\begin{align*}
y_1[n] &= \mathscr{Z}^{-1}\bigg\{Y_1(z)\bigg\}\\
       &= -\frac{8}{5}\mathscr{Z}^{-1}\bigg\{\frac{z}{z-2}\bigg\} 
          +\frac{1}{35}\mathscr{Z}^{-1}\bigg\{\frac{z}{z+\frac{1}{2}}\bigg\} 
          +\frac{18}{7}\mathscr{Z}^{-1}\bigg\{\frac{z}{z-3}\bigg\} \\
       &= -\frac{8}{5}2^n + \frac{1}{35}\left(-\frac{1}{2}\right)^n +\frac{18}{7}3^n
\end{align*}

Γνωρίζουμε πως,
\begin{gather*}
Y_2(z) = \frac{2z^4}{(-2+z)(-1+z)^2(-1+2z)} \\
\frac{Y_2(z)}{z} = \frac{2z^3}{(-2+z)(-1+z)^2(-1+2z)} \\
\frac{Y_2(z)}{z} = -4 \frac{1}{z-1} - \frac{2}{6}\cdot\frac{1}{z-\frac{1}{2}} -2 \frac{1}{(z-1)^2} + \frac{16}{3}\cdot\frac{1}{z-2}\\
Y_2(z) = -4 \frac{z}{z-1} - \frac{2}{6}\cdot\frac{z}{z-\frac{1}{2}} -2 \frac{z}{(z-1)^2} + \frac{16}{3}\cdot\frac{z}{z-2}
\end{gather*}

Επομένως,
\begin{align*}
y_2[n] &= \mathscr{Z}^{-1}\bigg\{Y_2(z)\bigg\} \\
       &= -4\mathscr{Z}^{-1}\bigg\{\frac{z}{z-1}\bigg\} 
          -\frac{2}{6}\mathscr{Z}^{-1}\bigg\{\frac{z}{z-\frac{1}{2}}\bigg\}
          -2\mathscr{Z}^{-1}\bigg\{\frac{z}{(z-1)^2}\bigg\}
          +\frac{16}{3}\mathscr{Z}^{-1}\bigg\{\frac{z}{z-2}\bigg\} \\
       &= -4u[n] - \frac{2}{6}\left(\frac{1}{2}\right)^n - 2n + \frac{16}{3}2^n
\end{align*}

\answer
Γνωρίζουμε πως,
\begin{equation*}
Z_1(z) = \frac{1}{1-\frac{1}{3}z^{-1}}, |z| < \frac{1}{3}
\end{equation*}

Επειδή $|z| < \frac{1}{3}$ εκτελούμε επαναλαμβανόμενη διαίρεση,

\begin{equation*}
  \setlength\extrarowheight{2pt}
  \begin{array}{r|l}
    \dropsign{-} 1\hphantom{{}+{}0z{}+{}0z^2{}+{}00z^3}& -\frac{1}{3}z^{-1} + 1 \\[2pt] \cline{2-2}
    1 - 3z\hphantom{{}+{}0z^2{}+{}00z^3}& -3z - 9z^2 - 27z^3 - \dots \\ \longdivline
    \dropsign{-} 3z\hphantom{{}-{}0z^2{}+{}00z^3}&\\
                 3z - 9z^2\hphantom{{}+{}00z^3}&\\ \longdivline
    \dropsign{-} 9z^2\hphantom{{}-{}00z^3}&\\
                 9z^2 + 27z^3&\\ \longdivline
                        \dots&
  \end{array}
\end{equation*}

Παρατηρούμε πως,
\begin{align*}
Z_1(z) &= \sum_{n=1}^{\infty} -3^nz^n \\
       &= \sum_{n=-\infty}^{\infty} -3^nz^nu[n-1] \\
       &= \sum_{n=-\infty}^{\infty} -\left(\frac{1}{3}\right)^n u[-n-1] z^{-n}
\end{align*}

Επομένως,
\begin{equation*}
z_1[n] = -\left(\frac{1}{3}\right)^nu[-n-1]
\end{equation*}

Γνωρίζουμε πως,
\begin{equation*}
Z_2(z) = \frac{1}{1-\frac{1}{3}z^{-1}}, |z| > \frac{1}{3}
\end{equation*}

Επειδή $|z| > \frac{1}{3}$ εκτελούμε επαναλαμβανόμενη διαίρεση,

\begin{equation*}
  \setlength\extrarowheight{2pt}
  \begin{array}{r|l}
    \dropsign{-} 1\hphantom{{}+{}\frac{0}{1}z^{-1}{}+{}\frac{0}{1}z^{-2}{}+{}\frac{0}{01}z^{-3}}& 1 -\frac{1}{3}z^{-1} \\[2pt] \cline{2-2}
    1 - \frac{1}{3}z^{-1}\hphantom{{}+{}\frac{0}{1}z^{-2}{}+{}\frac{0}{01}z^{-3}}& 1 + \frac{1}{3}z^{-1} + \frac{1}{9}z^{-2}  + \dots \\[2pt] \longdivline
    \dropsign{-} \frac{1}{3}z^{-1}\hphantom{{}-{}\frac{0}{1}z^{-2}{}+{}\frac{0}{01}z^{-3}}&\\[2pt]
                 \frac{1}{3}z^{-1} - \frac{1}{9}z^{-2}\hphantom{{}+{}\frac{0}{01}z^{-3}}&\\[2pt] \longdivline
    \dropsign{-} \frac{1}{9}z^{-2}\hphantom{{}-{}\frac{0}{01}z^{-3}}&\\[2pt]
                 \frac{1}{9}z^{-2} - \frac{1}{27}z^{-3}&\\[2pt] \longdivline
                        \dots&
  \end{array}
\end{equation*}

Παρατηρούμε πως,
\begin{align*}
Z_2(z) &= \sum_{n=0}^{\infty} \left(\frac{1}{3}\right)^nz^{-n} \\
       &= \sum_{n=-\infty}^{\infty} \left(\frac{1}{3}\right)^nu[n]z^{-n}
\end{align*}

Επομένως,
\begin{equation*}
z_2[n] = \left(\frac{1}{3}\right)^nu[n]
\end{equation*}

\end{answers}

\problem{Άσκηση 2}

\begin{questions}

\question{1} 
Αποδείξτε ότι αν ο μετασχηματισμός (\en{Laplace/Fourier}) της κρουστικής ενός συστήματος είναι μία
ρητή συνάρτηση δύο πολυωνύμων, τότε το σύστημα περιγράφεται από μία διαφορική εξίσωση.

\question{2} 
Αν το σύστημα έχει απόκριση $H(s) = \frac{A}{s+c}$, αποδείξτε ότι η διαφορική εξίσωση είναι $\frac{dy(t)}{dt} + cy(t) = Ax(t)$

\question{3}
Για να προσεγγίσουμε την παράγωγο $\frac{dy(t)}{dt}$, χρησιμοποιούμε τον ορισμό 
$\frac{dy(t)}{dt} = \lim_{T\to 0}\frac{y(t) - y(t-T)}{T} \approx \frac{y(t) - y(t-T)}{T}$, για $T$ πολύ μικρός αριθμός. 
Εκτιμήστε προσεγγιστικά την μορφή της διαφορικής εξίσωσης για $t = nT_s$, όπου $T_s$ πολύ μικρός αριθμός.

\question{4}
Για $n[n] = \equiv x(nT_s)$, δώστε την εξίσωση διαφορών που προκύπτει από την προσέγγιση του (3) και υπολογίστε την $H(z)$.

\question{5}
Δείξτε ότι $H(z) = \left.H(s)\right|_{s=\frac{10z^{-1}}{T}}$, υπάρχει δηλαδή διγραμμικός μετασχηματισμός μεταξύ
\en{Laplace} \& $\mathscr{Z}$. Σε ποιο πεδίο του $\mathscr{Z}$ αντιστοιχείται η περιοχή $Real(s) < 0$ ? Εξαρτάται η
ευστάθεια του διακριτοποιημένου συστήματος από το $T$ αν το αναλογικό σύστημα είναι ευσταθές?

\end{questions}

\solution

\begin{answers}

\answer
Γνωρίζουμε πως,
\begin{gather*}
\mathscr{F}\{h(t)\} = H(\omega) = \frac{Y(\omega)}{X(\omega)} = \frac{P(\omega)}{Q(\omega)} \\
\frac{Y(\omega)}{X(\omega)} =  \frac{a_0 + a_1\omega + a_2\omega^2 + \dots + a_n\omega^n}{
                                     b_0 + b_1\omega + b_2\omega^2 + \dots + b_m\omega^m} \\
\sum_{k=0}^{m} Y(\omega)b_k\omega^k = \sum_{k=0}^{n}X(\omega)a_k\omega^k \stepcounter{equation} \tag{\theequation} \label{eqn:diff}
\end{gather*}

Εφαρμόζοντας τον αντίστροφο μετασχηματισμό \en{Fourier} στην (\ref{eqn:diff}) έχουμε,

\begin{gather*}
\mathscr{F}^{-1}\big\{\sum_{k=0}^{m} Y(\omega)b_k\omega^k \bigg\} = \mathscr{F}^{-1}\bigg\{\sum_{k=0}^{n}X(\omega)a_k\omega^k \bigg\} \\
\sum_{k=0}^{m}\frac{b_k}{j^k}\frac{d^ky(t)}{dt^k} = \sum_{k=0}^{n}\frac{a_k}{j^k}\frac{d^kx(t)}{dt^k} 
\end{gather*}

Αντίστοιχα, για τον αντίστροφο μετασχηματισμό \en{Laplace} έχουμε,

\begin{gather*}
\mathscr{L}^{-1}\big\{\sum_{k=0}^{m'} Y(s)b_ks^k \bigg\} = \mathscr{L}^{-1}\bigg\{\sum_{k=0}^{n'}X(s)a_ks^k \bigg\} \\
\sum_{k=0}^{m'}b_k\frac{d^{k}y(t)}{dt^{k}} = \sum_{k=0}^{n'}a_k\frac{d^{k}x(t)}{dt^k} 
\end{gather*}

Και στις δύο περιπτώσεις το σύστημα μας περιγράφεται από μία διαφορική εξίσωση.


\answer
Έχουμε,
\begin{gather*}
H(s) = \frac{A}{s+c} \\
\frac{Y(s)}{X(s)} = \frac{A}{s+c} \\
sY(s) + cY(s) = AX(s) \\
\mathscr{L}^{-1}\bigg\{sY(s) + cY(s)\bigg\} = \mathscr{L}^{-1}\bigg\{AX(s)\bigg\}\\
\frac{dy(t)}{dt} + cy(t) = Ax(t)
\end{gather*}

\answer
Έχουμε,
\begin{gather*}
\frac{dy(t)}{dt} + cy(t) = Ax(t) \\
\frac{y(t) - y(t-T)}{T} + cy(t) = Ax(t) \quad \text{(προσεγγιστικά)}\\
y(t) - y(t-T) + cTy(t) = ATx(t) \\
y(nTs) - y(nTs - T) + cTy(nTs) = ATx(nTs) \quad \text{ για } t = nTs\\
y(nTs) - y(Ts(n-1)) + cTy(nTs) = ATx(nTs) \quad \text{ για } T \approx nTs
\end{gather*}

\answer

Έχοντας $x[n] = x(nTs)$ και $y[n] = y(nTs)$, τότε σύμφωνα με την (3):

\begin{equation*}
y[n] - y[n-1] + cTy[n] = ATx[n]
\end{equation*}

Εφαρμόζοντας αμφίπλευρο μετασχηματισμό $\mathscr{Z}$,

\begin{gather*}
\mathscr{Z}\bigg\{y[n] - y[n-1] + cTy[n]\bigg\} = AT\mathscr{Z}\bigg\{x[n]\bigg\} \\
(1-z^{-1})Y(z) + cTY(z) = ATX(z)\\
H(z) = \frac{Y(z)}{X(z)} = \frac{AT}{1-z^{-1}+cT} 
\end{gather*}

\answer
Γνωρίζουμε πως,

\begin{equation*}
H(s) = \frac{A}{s+c}
\end{equation*}

Επομένως, 
\begin{equation*}
\left.H(s)\right|_{s=\frac{1-z^{-1}}{T}} = \frac{A}{\frac{1-z^{-1}}{T} + c} = \frac{AT}{1-z^{-1}+cT} = H(z)
\end{equation*}

Και άρα υπάρχει διγραμμικός μετασχηματισμός μεταξύ $\mathscr{Z}$ και $\mathscr{L}$

\end{answers}

\problem{Άσκηση 3}

Έστω σύστημα $\Sigma_1$ Γ.Χ.Α αιτιατό στο οποίο δίνουμε είσοδο:
\begin{equation}
x[n]=-\frac{1}{3}(\frac{1}{2})^nu[n]-\frac{4}{3}2^nu[-n-1]
\end{equation}
Ο μετασχηματισμός $\mathscr{Z}$ της εξόδου είναι:
\begin{equation}Y(z) = \frac{1+z^{-1}}{(1-2z^{-1})(1-z^{-1})(1+\frac{1}{2}z^{-1})}\end{equation}

\begin{questions}

\question{1} Υπολογίστε τον $X(z)$ και βρείτε την \en{ROC} του μετασχηματισμού.
\question{2} Υπολογίστε την \en{ROC} στο $Y(z)$. Είναι το σύστημα ευσταθές?
\question{3} Υπολογίστε την απόκριση του $\Sigma_1$ σε συχνότητα/χρόνο $H(z)$, $h[n]$.
\question{4} Ποια είναι η εξίσωση διαφορών που ορίζει το σύστημα?
\question{5} Αντίστροφο καλείται το $\Sigma_2$, για το οποίο ισχύει ότι $\Sigma_2[\Sigma_1[x(t)]] = x(t)$. 
Υπολογίστε την απόκριση συχνότητας και την κρουστική απόκριση $G(z),g[n]$ του αντίστροφου συστήματος. Πριν τον 
υπολογισμό του $g[n]$, υπολογίστε το $g[0]$ χρησιμοποιώντας το $G(z)$. Επιβεβαιώστε το τελικό αποτέλεσμα.
\question{6} Έστω ότι συνδέουμε σειριακά το $\Sigma_1$ με σύστημα $\Sigma_3$ : $M(z) = \frac{1-z^{-1}}{1+z^{-1}}$. 
Για το νέο σύστημα $\Sigma_4$, υπολογίστε την απόκριση συχνότητας $F(z)$, την συχνότητα $\Omega$ όπου μεγιστοποιείται 
το πλάτος της απόκρισης. Το σύστημα είναι βαθυπερατό ή υψιπερατό?
\question{7} Υπολογίστε το \en{DFT} του $r[n] = (f[n] - \frac{1}{4}f[n-2])$.

\end{questions}

\solution

\begin{answers}

\answer 
Γνωρίζουμε ότι,

\begin{gather*} 
\mathscr{Z}\bigg\{\left(\frac{1}{2}\right)^{n}u[n]\bigg\} = \frac{1}{1-\frac{1}{2}z^{-1}}, \quad \text{\en{ROC: }}|z| > \frac{1}{2} \\
\mathscr{Z}\bigg\{-2^nu[-n-1]\bigg\} = \frac{1}{1-2z^{-1}}, \quad \text{\en{ROC: }}|z| < 2
\end{gather*}

Επομένως,

\begin{align*}
X(z) &= -\frac{1}{3}\mathscr{Z}\bigg\{\left(\frac{1}{2}\right)^nu[n]\bigg\} + \frac{4}{3}\mathscr{Z}\bigg\{-2^nu[-n-1]\bigg\}
     &= -\frac{1}{3}\cdot\frac{1}{1-\frac{1}{2}z^{-1}} + \frac{4}{3}\cdot\frac{1}{1-2z^{-1}}, \quad \text{\en{ROC: }} \frac{1}{2} < |z| < 2
\end{align*}

\answer
Γνωρίζουμε ότι,

\begin{gather*} 
X(z) = \frac{z^2}{\left(z-\frac{1}{2}\right)(z-2)} \\
Y(z) = \frac{z^2(z+1)}{(z-2)(z-1)\left(z+\frac{1}{2}\right)}
\end{gather*}

Άρα,
\begin{gather*}
H(z) = \frac{Y(z)}{X(z)} \\
H(z) = \frac{\left(z-\frac{1}{2}\right)(z+1)}{\left(z+\frac{1}{2}\right)(z-1)}
\end{gather*}

Και επειδή το σύστημα είναι αιτιατό, $|z| > r$ αφού πρέπει $z \to \infty$. Εξαιτίας των πόλων έχουμε $|z| > 1$.
Τελικά \en{ROC: } $1 < |z| < 2$ και το σύστημα είναι ευσταθές αφού ο μοναδιαίος κύκλος αποτελεί όριο του πεδίου σύγκλισης.


\answer

Γνωρίζουμε πως,
\begin{align*}
H(z) &= \frac{\left(z - \frac{1}{2}\right)(z+1)}{\left(z+\frac{1}{2}\right)(z-1)} \\
     &= \frac{1}{3}\cdot\frac{z}{z+\frac{1}{2}}+\frac{2}{3}\cdot\frac{z}{z-1} + \frac{1}{2}\cdot\frac{1}{z+\frac{1}{2}}
\end{align*}


Επομένως,

\begin{equation*}
h[n] = \mathscr{Z}^{-1} \bigg\{ H(z)\bigg\} = \frac{1}{3}\left(-\frac{1}{2}\right)^n u[n] + \frac{2}{3}u[n] -\left(-\frac{1}{2}\right)^n u[n-1]
\end{equation*}


\answer

Έχουμε,

\begin{gather*}
H(z) = \frac{Y(z)}{X(z)} \\
H(z) = \frac{\left(z-\frac{1}{2}\right)(z+1)}{\left(z+\frac{1}{2}\right)(z-1)} \\
\left(z-\frac{1}{2}\right)(z+1)X(z) = \left(z+\frac{1}{2}\right)(z-1)Y(z) \\
X(z) + \frac{1}{2}X(z)z^{-1} - \frac{1}{2}z^{-2}X(z) = Y(z) - \frac{1}{2}z^{-1}Y(z) - \frac{1}{2}z^{-2}Y(z)\\
\mathscr{Z}^{-1}\bigg\{X(z) + \frac{1}{2}X(z)z^{-1} - \frac{1}{2}z^{-2}X(z)\bigg\} = 
\mathscr{Z}^{-1}\bigg\{Y(z) - \frac{1}{2}z^{-1}Y(z) - \frac{1}{2}z^{-2}Y(z)\bigg\} \\
x[n] + \frac{1}{2}x[n-1] - \frac{1}{2}x[n-2] = y[n] - \frac{1}{2}y[n-1] - \frac{1}{2}y[n-2]
\end{gather*}

\answer
Για την νέα απόκριση συχνότητας $H'(z)$ έχουμε,

\begin{gather*}
H'(z) = \frac{X(z)}{Y(z)} \\
H'(z) = \frac{\left(z+\frac{1}{2}\right)(z-1)}{\left(z-\frac{1}{2}\right)(z+1)}\\
H'(z) = \frac{z^2-\frac{1}{2}z-\frac{1}{2}}{z^2+\frac{1}{2}z-\frac{1}{2}}
\end{gather*}

Εφαρμόζουμε τη μέθοδο της διαίρεσης πολυωνύμων,
\begin{equation*}
  \setlength\extrarowheight{2pt}
  \begin{array}{r|l}
    \dropsign{-} z^2 - \frac{1}{2}z-\frac{1}{2}\hphantom{{}+{}\frac{0}{1}z^{-1}}& z^2+\frac{1}{2}z-\frac{1}{2} \\[2pt] \cline{2-2}
    z^2+\frac{1}{2}z-\frac{1}{2}\hphantom{{}+{}\frac{0}{1}z^{-1}}& 1 - z^{-1} + \dots \\[2pt] \longdivline
     \dropsign{-} -z\hphantom{{}-{}\frac{0}{1}{}+{}\frac{0}{1}z^{-1}}&\\[2pt]
                  -z-\frac{1}{2}+\frac{1}{2}z^{-1}&\\[2pt] \longdivline
                     \frac{1}{2}-\frac{1}{2}z^{-1}&\\[2pt]
  \end{array}
\end{equation*}

Άρα $h'[0] = 1$

Αναλύοντας το $H'(z)$ σε κλάσματα παίρνουμε:

\begin{gather*}
Η'(z) = \frac{1}{3}\cdot\frac{z}{z-\frac{1}{2}} + \frac{2}{3}\cdot\frac{z}{z+1} - \frac{1}{2}\cdot\frac{1}{z-\frac{1}{2}}\\
\mathscr{Z}^{-1}\bigg\{H'(z)\bigg\} = \mathscr{Z}^{-1}\bigg\{\frac{1}{3}\cdot\frac{z}{z-\frac{1}{2}} + \frac{2}{3}\cdot\frac{z}{z+1} - \frac{1}{2}\cdot\frac{1}{z-\frac{1}{2}}\bigg\}\\
h'[n] = \frac{1}{3}\left(\frac{1}{2}\right)^nu[n] + \frac{2}{3}(-1)^nu[n] - \left(\frac{1}{2}\right)^nu[n-1]
\end{gather*}

\end{answers}

\end{document}

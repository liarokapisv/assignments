\newpage\subsection*{Άσκηση 1}

Έστω τα πεπερασμένα σήματα διακριτού χρόνου 
\begin{align*}
x[n] &= \delta[n] + 2\delta[n-1] + \delta[n-1] + 3\delta[n-3] + \delta[n-5] + 2\delta[n-6] - 4\delta[n-7] \\
h[n] &= \delta[n] - 2\delta[n-1] + d[n-2]
\end{align*}

\begin{enumerate}[label=\textbf{(\alph*)}]
    \item Αν $X[k], H[k]$ είναι οι 8-σημείων \en{DFT} των σημάτων $x[n],h[n]$ και $Y[k] = X[k]H[k]$, να βρείτε τις
        τιμές του σήματος $y[n]$ που προκύπτει με ένα 8-σημείων αντίστροφο \en{DFT} του $Y[k]$.
        Εξηγείστε.

    \item Να σχεδιάσετε τα σήματα $x[n],h[n]$ και $y[n]$
    \item Αν επαναλάβετε το (α) με \en{DFT} Ν σημείων, να βρείτε την τιμή του $N$ ώστε $y[n] = x[n] \ast h[n]$ 
        για $n = 0, 1, ..., N -1$. Εξηγείστε.
    \item Με βάση τον μετασχηματισμό $X[k]$, ορίζουμε τις ακολουθίες
        \begin {align*}
        P[k] &= j^kX[k], k = 0, ..., 7 \\
        Q[k] &= Re\{X[2k]\}, k = 0, 1, 2, 3
        \end {align*}
        ως τους \en{DFT} των σημάτων $p[n]$ και $q[n]$, αντίστοιχα. Χωρίς να υπολογίσετε τους ευθείς και αντίστροφους 
        \en{DFT} των σχετικών ακολουθιών, αλλα χρησιμοποιώντας μόνο τις ιδιότητες του \en{DFT}:
        \begin{enumerate}[label=\textbf{(δ.\arabic*)}]
            \item Να βρείτε αναλυτικά και να σχεδιάσετε το σήμα $p[n]$. Εξηγείστε.
            \item Να βρείτε αναλυτικά και να σχεδιάσετε το σήμα $q[n]$. Εξηγείστε.
        \end{enumerate}
\end{enumerate}

\subsubsection*{Λύση}

Έχουμε,
\begin{align*}
    x &= \{1,2,-1,3,0,-1,2,-2\} \\
    y &= \{1,-2,1,0,0,0,0,0,0\}
\end{align*}

\begin{enumerate}[label=\textbf{(\alph*)}]
\item
    Μπορούμε να υπολογίσουμε την γραμμική συνέλιξη υπολογίζοντας πρώτα την γραμμική συνέλιξη. \\
    Για τη γραμμική συνέλιξη έχουμε,
        \begin{equation*} y_l = \sum_{m=0}^{7} x[n]h[n-m] \end{equation*}
    και πιο συγκεκριμένα,
    \begin{align*}
        y_l[0] &= x[0]h[0] = 1 \\
        y_l[1] &= x[0]h[1] + x[1]h[0] = 0 \\
        y_l[2] &= x[0]h[2] + x[1]h[1] + x[2]h[2] = -4 \\
        y_l[3] &= x[0]h[3] + ... + x[3]h[0] = 7 \\
        y_l[4] &= x[0]h[4] + ... + x[4][h1] = -7 \\
        y_l[5] &= x[0]h[5] + ... + x[5]h[0] = 2 \\
        y_l[6] &= x[0]h[6] + ... + x[6]h[0] = 4 \\
        y_l[7] &= x[0]h[7] + ... + x[7]h[0] = -9 \\
        y_l[8] &= x[0]h[8] + ... + x[8]h[0] = 10 \\
        y_l[9] &= x[0]h[9] + ... + x[9]h[0] = -4 \\
    \end{align*}
    Τώρα μπορούμε να υπολογίσουμε τη κυκλική με το γνωστό τρόπο:
        \begin{equation*} y = \{11, -4, -4, 7, -7, 2, 4, -9, 10, -4\} \end{equation*}

\item

\end{enumerate}

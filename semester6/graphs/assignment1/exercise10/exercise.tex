\newpage\subsection*{Άσκηση 10}

Έστω ένα απλό γράφημα $G$ με $n$ κορυφές και έστω $k$ με $1 < k < n-1$. Άν όλα 
τα επαγώμενα υπογραφήματα του $G$ με $k$ κορυφές έχουν το ίδιο πλήθος ακμών τότε το $G$ 
είναι είτε το πλήρες γράφημα με $n$ κορυφές είτε το κενό γράφημα με $n$ κορυφές.

\subsubsection*{Λύση}

Η αντίστροφη φορά εύκολα αποδεικνύεται.

Για να αποδείξω την κανονική φορά, πρώτα θα δείξω πως το $G$ και όλα τα επαγώμενα υπογραφήματα $n-1$ κόμβων είναι κανονικά.

Για κάθε υπογράφημα με κορυφές $\{v_1,v_2,..,v_k,v_{k+1}\}$ μπορώ να διαλέξω τυχαία έναν κόμβο $v_i$. Το πλήθος των ακμών $E_0$ μεταξύ των υπολοίπων κόμβων του υπογραφήματος
θα είναι σταθερός εξαιτίας των συνθηκών. Συνολικά στο υπογράφημα θα έχουμε $E_0 + d(v_i)$ ακμές. Ωστόσο το πλήθος των ακμών μέσα σε αυτό το υπογράφημα δεν αλλάζει ανεξάρτητα απο ποιο
$v_i$ διαλέξω και καθώς το $E_0$ είναι σταθερό για όλα τα υπογραφήματα $k$ κορυφών, όλοι οι κόμβοι στο υπογράφημα θα έχουν τον ίδιο βαθμό $\lambda$. Άρα θα ισχύει πως 
$2(E_0 + \lambda) = \sum d(v) = \sum \lambda = (k+1)\lambda \Rightarrow \lambda = \frac{2E_0}{k-1}$. Αυτό υποδηλώνει πως όλα τα επαγώμενα υπογραφήματα $k+1$ κόμβων θα είναι κανονικα γραφήματα ίδιου
βαθμού και επομένως θα έχουν και ίσο αριθμό ακμών. Επαγωγικά αποδεικνύουμε την παραπάνω πρόταση. 

Κάθε κόμβος του γράφου θα έχει βαθμό $r$.
Όλα τα υπογραφήματα $n-1$ κόμβων θα έχουν βαθμό $r'$ για τον οποίον θα ισχύει είτε $r' = r$ είτε $r' = r-1$.
Αν $r' = r$ τότε ο κόμβος που περισσεύει δε θα μπορεί να συνδεθεί με κανέναν άλλο κόμβο καθώς ο βαθμός του θα ξεπέρναγε το $r$. 
Σε αυτήν την περίπτωση θα πρέπει $r = r' = 0$ και επομένως υποχρεωτικά ο γράφος θα πρέπει να είναι ο κενός.
Αν $r' = r-1$ τότε ο κόμβος που περισσεύει θα πρέπει να συνδεθεί με όλους τους αλλους $n-1$ κόμβους διαφορετικά θα έμενε κόμβος που δεν
θα είχε βαθμό $r$. Έτσι υποχρεωτικά $r=n-1$ και ο γράφος είναι ο πλήρης $n$ κορυφών.

Έτσι ο γράφος είναι υποχρεωτικά είτε κενός είτε πλήρης. 

\newpage\subsection*{Άσκηση 5}

Ορίζουμε ως γέφυρα ενός συνεκτικού γραφήματος $G$ μία ακμή $e \in E(G)$ για την οποία ισχύει ότι 
το $G\\e$ δεν είναι συνεκτικό. Δείξτε ότι ένα απλό κανονικό συνεκτικό διμερές γράφημα με βαθμό τουλάχιστον
2, δεν περιέχει γέφυρα.

\subsubsection*{Λύση}

Έστω πως υπάρχει γέφυρα $e \in E(G)$ αφαιρώντας την οποία σπάμε τον γράφο σε δύο συνιστώσες $V_1$ και $V_2$.
Η $V_1$ είναι διμερές γράφος που αποτελείται απο δύο σύνολα διαμέρισης $X_1$ και $X_2$ ενώ η $e$ ήταν προσπίπτουσα
ακμή κόμβου που άνηκε στον $X_1$. Είναι γνωστό πως επειδή ο $V_1$ 
είναι διμερής, συνεκτικός και σχεδόν κανονικός γράφος (όλοι οι κόμβοι εκτός απο τον κόμβο στον οποίο η $e$ ήταν προσπίπτουσα έχουν ίδιο βαθμό), 
θα έχουμε $|V(X_1)| = |V(X_2)|$ ενώ θα ισχύει επίσης πως οι προσπίπτουσες στον $X_1$ να είναι κατα ένα μικρότερες απο τις προσπίπτουσες στον $X_2$. Αυτή η διαφορά στον αριθμό
των προσπιπτουσών σημαίνει πως μία ακμή του $X_2$ θα πρέπει να συνδέεται με έναν κόμβο του $V_2$ το οποίο είναι άτοπο. Άρα δεν υπάρχει γέφυρα.

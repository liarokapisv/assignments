\newpage\subsection*{Άσκηση 7}

Έστω $A$ ο πίνακας γειτνίασης ενός γραφήματος $G$ με $n$ κορυφές. 

\begin{enumerate}[i.]

\item
Δείξτε ότι για οποιοδήποτε ζεύγος από δείκτες $i$ και $j$ με $1 \le i,j \le n$ το $(i,j)$ στοιχείο του πίνακα $A^l$ όπου
$1 \le l \le n$ είναι ίσο με τον αριθμό των μεταξύ τους διαφορετικών $(v_i,v_j)$ περιπάτων μήκους $l$ στο $G$.

\item
Έστω $Y = A + A^2 + ... + A^{n-1}$. Αν κάποιο μη διαγώνιο στοιχείο του $Y$ είναι 0, τότε τι συμπεραίνετε για το γράφημα $G$?

\item
Έστω τετραγωνικός πίνακας $M$. Σύμβολίζουμε με $Tr(M)$ το ίχνος του πίνακα, δηλαδή το άθροισμα των διαγώνιων στοιχείων του. 
        Δείξτε ότι αν το γράφημα $G$ δεν έχει βρόγχους, τότε το πλήθος των τριγώνων στο $G$ ισούται με $\frac{Tr(A^3)}{6}$.

\end{enumerate}

\subsubsection*{Λύση}

\begin{enumerate}[i.]

\item
Για $l=1$ προφανώς ισχύει. Έστω πως ισχύει για $l=m$, τότε:
\begin{equation*}
A^{m+1}_{ij} = \sum_{1 \le k \le n} A^m_{ik}*A_{kj} 
\end{equation*}
To πλήθος των μονοπατιών $(v_i, v_j)$ μήκους $m+1$ ισούτε με το πλήθος των μονοπατιών $(v_i,v_k)$ μήκους $m$ για τα οποία
υπάρχει η ακμή $v_kv_j$. Αυτός ο αριθμός ωστόσο συμπίπτει με το παραπάνω άθροισμα γινομένων και επομένως θα ισχύει και για $l=m+1$. 
Επαγωγικά θα ισχύει για όλα τα $l \ge 1$
        
\item
Όπως φαίνεται και απο το προηγούμενο ερώτημα κάθε στοιχείο $Y_{ij}$ θα εκφράζει το πλήθος όλων των μονοπατιών $(v_i, v_j)$.
Αν κάποιο στοιχείο είναι 0 τότε οι αντίστοιχοι κόμβοι δε συνδέονται μεταξύ τους και επομένως ο γράφος δεν είναι συνεκτικός.

\item
Κάθε στοιχείο $A^3_{ii}$ της διαγωνίου υποδηλώνει τον αριθμό των μονοπατιών μήκους 3 που αρχίζουν απο και τελειώνουν στον κόμβο $v_i$.
Προστίθοντας τα παίρνουμε το σύνολο όλων των κύκλων μήκους 3. Για να πάρουμε τον αριθμό των τριγώνων θα πρέπει πρώτα να διαιρέσουμε
με το 2 για να μην ξαναμετρήσουμε ίδιους κύκλους διαφορετικής φορας, αλλα και με το 3 διότι ο ίδιος κύκλος θα ξαναμετρηθεί για κάθε ένα απο
        τους 3 κόμβους ενός κύκλου. Έτσι ο συνολικός αριθμός τριγώνων είναι $\frac{Tr(A^3)}{6}$
\end{enumerate}



\newpage\subsection*{Άσκηση 6}

Ένα σύνολο ανεξάρτητων κορυφών είναι ένα σύνολο απο κορυφές του γραφήματος οι οποίες δεν ενώνονται μεταξύ τους με καμία ακμή. 
Συμβολίζουμε με $\beta_0(G)$ το μέγιστο πλήθος ανεξάρτητων κορυφών του γραφήματος $G$. Δείξτε ότι αν το   $G$ είναι απλό και περιέχει 
τρίγωνο τότε $\Delta(G) \le \beta_0(G)$ και $|E(G)| \le \frac{|V(G)|\beta_0(G)}{2}$.

\subsubsection*{Λύση}

Ο κόμβος $v \in V(G), d(v) = \Delta(G)$ έχει $\Delta(G)$ γείτονες. Ωστόσο αυτοί οι γείτονες δεν πρέπει να συνδέονται
αφου τότε θα δημιουργόταν τρίγωνο. Άρα οι γείτόνες του $v$ αποτελούν ένα ανεξάρτητο σύνολο μεγέθους $\Delta(G)$. Άρα
$\Delta(G) \le \beta_0(G)$. Επίσης, 

\begin{equation*}
    2E(G) = \sum_{v \in V(G)} d(v) \le |V(G)|\Delta(G) \le |V(G)|\beta_0(G)
\end{equation*}

Επομένως
\begin{equation*}
    E(G) \le \frac{|V(G)|\beta_0(G)}{2}
\end{equation*}

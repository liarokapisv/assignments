\newpage\subsection*{Άσκηση 3}
Κανένα φύλλο δε θα εμφανιστεί στον κώδικα \en{Prufer} αφού δεν εμφανίζεται
στον αναδρομικό υπολογισμό. Όλες οι κορυφές βαθμού $n$ θα πρέπει να εμφανιστούν $n-1$ φορές πριν γίνουν φύλλα
των γράφων των αναδρομικών υπολογισμών. Έτσι άμα υπάρχει κορυφή βαθμού 3 και πάνω, θα εμφανίζεται τουλάχιστον δύο φορές 
στον κώδικα \en{Prufer}. Επομένως οι κώδικες \en{Prufer} χωρίς διπλές κορυφές αντιπροσωπεύουν δέντρα - γραμμές με βαθμούς το πολύ 2.
Η σχέση προκύπτει απο το γεγονός πως υπάρχουν $ \frac{n(n-1)}{2} $ διακριτές τέτοιες γραμμές άμα λάβουμε υπόψιν τις συμμετρίες. 

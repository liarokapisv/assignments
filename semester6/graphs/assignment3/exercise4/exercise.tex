\newpage\subsection*{Άσκηση 4}
Ένα δέντρο θα έχει τα ελάχιστα σε αριθμό διακριτά μονοπάτια
αν οι κορυφές ζυγού βαθμού δεν αποτελούν άκρα μονοπατιού και οι
κορυφές περιττού βαθμού αποτελούν άκρα διακριτού μονοπατιού (διαφορετικά μπορούμε να ενώσουμε δύο διαφορετικά μονοπάτια).
Η σχέση επομένως θα ισχύει για απλά δέντρα αφού με $2k$ άκρα διακριτών μονοπατιών πρέπει να υπάρχουν $k$ διακριτά μονοπάτια στο δέντρο.
Έστω πως ένα δάσος έχει $2k$ περιττού βαθμού κορυφές και $n$ συνιστώσες. Κάθε συνιστώσα ως δέντρο πρέπει να έχει τουλάχιστον 2 κορυφές περιττού βαθμού.
Ενώνω ανα ζεύγη κορυφές περιττού βαθμού διαφορετικών συνιστωσών μέχρι να μείνει ένα δέντρο. Το δέντρο αυτό θα έχει $2k - 2(n-1) = 2(k-n+1)$ περιττές 
κορυφές και όπως αποδείχτηκε παραπάνω θα έχει τουλάχιστον $k-n+1$ διακριτά μονοπάτια. Το αρχικό δάσος αντί για το μονοπάτι που ενώνει τις παραπάνω συνιστώσες
θα έχει τα $n$ μονοπάτια που μαζί με τις προστεθόμενες ακμές το αποτελούν. Έτσι θα έχει $(k-n+1) - 1 + n = k$ μονοπάτια και επομένως θα ισχύει η σχέση.

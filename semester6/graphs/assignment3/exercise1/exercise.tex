\newpage\subsection*{Άσκηση 1}
1. Για τα δέντρα με 2 κορυφές η σχέση ισχύει. Έστω πως ισχύει για όλα τα δέντρα $k$ κορυφών.
Για ένα οποιοδήποτε δέντρο $k+1$ κορυφών $G$, αφαιρώ ένα φύλλο $l$ και παίρνω ένα δέντρο $k$ κορυφών $G'$. 
Ονομάζουμε τον προσπίπτων $l$ κόμβο ως $t$. Αν ο $t$
είναι φύλλο τότε ο $G$ δεν θα έχει περισσότερα φύλλα και η σχέση θα ισχύει και για τον $G$ αφού η ακμή
$t-l$ δεν αυξάνει τον αριθμό των κόμβων με βαθμό μεγαλύτερο του 2.
Aν o $t$ δεν είναι φύλλο τότε λόγω της ακμής $t-l$ το $t$ θα συνεισφέρει 1 περισσότερο στο άθροισμα, ενώ 
τα φύλλα θα αυξηθούν κατα ένα. Έτσι θα ισχύει η σχέση και στο $G$. 
Επομένως λόγω επαγωγής η σχέση θα ισχύει σε κάθε δέντρο με τουλάχιστον 2 κορυφές.

2. Η σχέση προκύπτει απευθείας παραγοντοποιώντας τους βαθμούς στο άθροισμα μαζί με το ότι το $n_1$ είναι ίσο με τον αριθμό των φύλλων.

\newpage\subsection*{Άσκηση 6}
Αρχικά, απο την προηγούμενη άσκηση γίνεται φανερό πως η μεγαλύτερη βαρυκεντρότητα θα ανοίκει σε φύλλο 
(διαφορετικά θα υπήρχε μία γειτονική κορυφή με μεγαλύτερη βαρυκεντρότητα).
Έπειτα θα δείξουμε πως από κάθε γράφο συγκεκριμένων κορυφών, το δέντρο-γραμμή έχει την κορυφή με την μεγαλύτερη βαρυκεντρότητα.
Για 2 κορυφές προφανώς ισχύει.
Έστω πως ισχύει για $k$ κορυφές. Έστω γράφος $G$ με $k+1$ κορυφές, φύλλο με τη μεγαλύτερη βαρυκεντρότητα $l$ και γείτονάς του $t$.
Θα πρέπει να ισχύει πως $s(l,G) = s(t,G/l) + |V(G/l)|$. 
Επομένως ο γράφος με τη μεγαλύτερη βαρυκεντρότητα θα είναι αυτός που μεγιστοποιεί τον όρο                                                               
$s(t,G/l)$ το οποίο θα συμβαίνει όταν ο $t$ είναι το ένα ακρο του δέντρου-γραμμής $k$ κορυφών. Επομένως ο γράφος $k+1$ με τη μεγαλύτερη βαρυκεντρότητα θα                                                       
είναι και αυτός δέντρο-γραμμή. Λόγω επαγωγής αποδείχτηκε το παραπάνω.                                                                                                                                           Για ένα γράφο-γραμμή $n$ κορυφών θα ισχύει πως η βαρυκεντρότητα του φύλλου θα είναι ίση με $\binom{n}{2}$.
Οποιαδήποτε άλλη κορυφή θα έχει μικρότερη βαρυκεντρότητα.
     


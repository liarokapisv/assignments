\subsection*{Άσκηση 8}

Δείξτε ότι αν το $G$ είναι απλό γράφημα με $n$ κορυφές και $\delta(G) \ge (n + k -2)/2$ με $k \ge 2$, τότε το $G$ είναι τουλάχιστον $k$-συνεκτικό.

\subsubsection*{Λύση}

Αφαιρώ $k-1$ κόρυφές και έχω $|V(G')| = n - k + 1$ και $\delta(G') \ge (n-k)/2$. 
Έστω πως το $G'$ δεν είναι συνεκτικό, τότε θα υπάρχουν κορυφές $u,v$ που δεν επικοινωνούν και συνεπώς
δεν έχουν και κοινούς γείτονες. Συνολικά λόγω του $\delta(G')$ οι γείτονές τους θα είναι τουλάχιστον $n-k$ 
και επομένως ο $G'$ θα έχει τουλάχιστον $n-k+2$ κορυφές. Άρα άτοπο και επομένως το $G'$ είναι συνεκτικό
και το $G$ τουλάχιστον $k$-συνεκτικό.

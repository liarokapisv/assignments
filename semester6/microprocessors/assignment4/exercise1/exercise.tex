\newpage\subsection*{Άσκηση 1}

Στο μ\en{LAB} να γραφτεί πρόγραμμα \en{Assembly} που να ελέγχει μέσω της διακοπής τύπου \en{RST} 6.5 τα φώτα ενός χώρου.
Όταν προκαλείται διακοπή \en{RST 6.5} να ανάβουν όλα τα \en{LED} της πόρτας εξόδου. Αυτό να παραμένει για περίπου ένα λεπτό 
της ώρας και μετά να σβήνει. Αν όμως ενδιάμεσα ξαναενεργοποιηθεί η διακοπή να ανανεώνεται ο χρόνος του ενός λεπτού. Ο χρόνος που
παραμένει στην κατάσταση αυτή να απεικονίζεται σε \en{sec} συνεχώς στα 2 αριστερότερα δεκαεξαδικά ψηφία των \en{7-segment displays} και
σε δεκαεξαδική μορφή.

Να γίνει χρήση των ρουτινών χρονοκαθυστέρησης του εκπαιδευτικού συστήματος μ\en{LAB}.

\subsubsection*{Λύση}

\selectlanguage{english}
\inputminted{text}{./exercise1/exercise.8085}
\selectlanguage{greek}


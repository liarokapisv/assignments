\newpage \subsection*{Άσκηση 3}

Στο μΕ 8085 εκτελείται η εντολή \en{CALL 3000H}. Ο μετρητής προγράμματος είναι \en{PC = 2000H} και ο δείχτης σωρού \en{SP = 4000H}. Στο μέσο της εκτέλεσης
της εντολής 	συμβαίνει διακοπή \en{RST 5.5}. Δώστε τις νέες τιμές των \en{PC, SP}, το περιεχόμενο του σωρού καθώς και τις λειτουργίες που συμβαίνουν
στν αρχή και στην επιστροφή απο την ρουτίνα εξυπηρέτησης.


\subsubsection*{Λύση}

Έστω πως η διακοπή συμβαίνει στο μέσον της εκτέλεσης της \en{CALL 3000H}, τότε ο \en{C} θα δείχνει στη θέση μνήμης \en{2000H}.
Άμα ενεργοποηθεί η διακοπή τότε ο μικροεπεξεργαστής θα περιμένει για να ολοκληρωθεί η εκτέλεση της \en{CALL 3000H}.
Μετέπειτα ο έλεγχος μεταφέρεται στην κατάλληλη ρουτίνα εξυπηρέτησης. Kαθώς ο έλεγχος περνάει στη ρουτίνα εξυπηρέτησης,
ο \en{PC} θα δείχνει στη διεύθυνση \en{3000H} η οποία και θα είναι αποθηκευμένη στο σωρό του συσσωρευτη με τα δύο περισσότερο 
σημαντικά ψηφία της στην προηγούμενη θέση του συσσωρευτή και με τα δύο λιγότερο σημαντικά στην προηγούμενη απο αυτή. 
Επομένως, η διεύθυνση στην οποία θα δείχνει ο δείκτης στοίβας θα είναι η παραπάνω διεύθυνση των δύο λιγότερο σημαντικών ψηφίων. 
Τελικά ο \en{PC} θα δείχνει στη διεύθυνση \en{0034H}, μιας και πρόκειται για διακοπή \en{RST 6.5}.

Βάσει αυτών, έχουμε:

\en{
PC: 0034H \\
SP: 3FFEH \\
(3FFFH) $\leftarrow$ 30H \\
(3FFEH) $\leftarrow$ 00H \\
}

Μετά την την εκτέλεση της ρουτίνας εξυπηρέτησης, ο δείκτης σωρού θα δείχνει ξανά στη διεύθυνση \en{4000H}, ενώ ο \en{PC} στη 
διεύθυνση \en{3000H}.

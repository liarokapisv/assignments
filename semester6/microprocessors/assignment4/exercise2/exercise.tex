\newpage \subsection*{Ασκηση 2}

Να υλοποιηθεί και να εκτελεστεί στο μ\en{LAB} πρόγραμμα σε \en{assembly} που όταν προκαλείται διακοπή τύπου \en{RST 6.5} να διαβάζει
τα 2 διαδοχικά δεκαεξαδικά ψηφία ενός αριθμού (0-255) που δίνονται στη συνέχεια απο το πλκτρολόγιο και να τα απεικονίζει στα 2 δεξιότερα
\en{7-segment display}. Να συγκρίνει την τιμή αυτή με δύο κατώφλια Κ1 και Κ2 με Κ1 < Κ2, που οι τιμές τους βρίσκονται στους καταχωρητές 
\en{B} και \en{C} αντίστοιχα. Στη συνέχεια να ανάβει ένα απο τα τρία \en{LED} εξόδου που αντιστοιχούν στις περιοχές τιμών $[0..K1],(K1..K2]$ και $(K2..FFH]$.

\subsubsection*{Λύση}

\selectlanguage{english}
\inputminted{text}{./exercise2/exercise.8085}
\selectlanguage{greek}

\newpage\subsection*{Άσκηση 2}

Να γραφεί σε \en{assembly} πρόγραμμα που να απεικονίζει ένα αναμμένο \en{led}
το οποίο να 
κινείται  αριστερά  (από  το  \en{LSB}  προς  το  \en{MSB})  και  να  συνεχίζει  να  κινείται  κυκλικά  (θέσεις  \en{led} 
0123456701... κ.λπ.) όταν το 
\en{MSB} της θύρας των \en{dip switch} είναι \en{OFF}. Αλλιώς, όταν το 
\en{MSB} των \en{dip switch} γίνεται \en{ON} να αναστρέφεται η κατεύθυνση και το αναμμένο \en{led} να κινείται δεξιά (από το 
\en{MSB} προς το \en{LSB}). Τέλος, όταν το \en{LSB} της θύρας των \en{dip switch} γίνεται \en{ON},
το \en{led} να σταματάει εκεί που βρίσκεται. 
Στη συνέχεια, όταν ξαναγίνει \en{OFF} να συνεχίζεται η κίνησή του σύμφωνα με το 
\en{MSB} των \en{dip switch}.
Να γίνει χρήση της θύρας εισόδου \en{dip switch} (θέση μνήμης 2000 
\en{Hex}) και της θύρας εξόδου των 
\en{LED} (που αντιστοιχεί στη θέση μνήμης 3000 \en{Hex}
–
προσοχή στην αντίστροφη λογική απεικόνισης). 
Στη συνέχεια, όταν ξαναγίνει ON να συνεχίζεται η κίνηση του αναμμένου \en{led}. 
(Διάρκεια ανάμματος $\sim 1/2$ \en{sec}). 

\subsubsection*{Λύση}

\selectlanguage{english}
\inputminted{text}{askisi2.8085}
\selectlanguage{greek}



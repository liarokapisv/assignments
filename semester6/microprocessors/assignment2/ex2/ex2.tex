\newpage \subsection*{Άσκηση 2}

Δίνεται ένα μΥ-Σ 8085 που ελέγχει τα \en{LED} της πόρτας εξόδου (3000Η) εξομοιώνοντας με αυτά τα φώτα ενός χώρου.
Να γραφτεί πρόγραμμα \en{Assembly}, που όταν το \en{MSB} της θύρας εισόδου \en{dip switch} (θέση μνήμης 2000Η) 
από \en{OFF} γίνει \en{ON} και ξανά \en{OFF} τότε να ανάβει όλα τα \en{LED} της πόρτας εξόδου. 
Αυτό να παραμένει ανοιχτό για περίπου 30 \en{sec} και μετά να σβήνει. Αν όμως ενδιάμεσα ξαναενεργοποιηθεί 
το \en{push-button (OFF-ON-OFF} το \en{MSB} των \en{dip switch)} να ανανεώνεται ο χρόνος των 30 \en{sec}. 
Να γίνει χρήση των ρουτινών χρονοκαθυστέρησης του εκπαιδευτικού συστήματος μ\en{LAB}. Θεωρήστε ότι το
σύστημα παρακολουθεί με διακριτική ικανότητα όχι μικρότερη του 1/2 \en{sec}.

\subsubsection*{Λύση}


\selectlanguage{english}
\inputminted{text}{./ex2/ex2.8085}
\selectlanguage{greek}
